\documentclass[master,14pt,subf,href,colorlinks=true
% ,times        % шрифт Times как основной
%,fixint=false % отключить прямые знаки интегралов
]{disser}

\usepackage[utf8]{inputenc}
\usepackage{graphicx}
\usepackage{mathtools}
\usepackage{url}
\usepackage{courier}
\usepackage{array}
%\usepackage{enumitem}
\usepackage{paralist}
\usepackage[inline]{enumitem}
\newcolumntype{P}[1]{>{\raggedright\arraybackslash}p{#1}}
\usepackage{hyperref}
\hypersetup{
     colorlinks   = false,
     citecolor    = gray
}

\newcommand{\ti}[1]{\foreignlanguage{English}{\textit{#1}}}
\newcommand{\fl}[1]{\foreignlanguage{English}{#1}}
\setlist[enumerate,1]{label=\textit{\alph*)}}



\usepackage[a4paper, mag=1000, includefoot, left=3cm, right=2cm, top=2cm, bottom=2cm, headsep=1cm, footskip=1cm]{geometry}
\usepackage[T2A]{fontenc}
\usepackage[english,russian]{babel}
\ifpdf\usepackage{epstopdf}\fi

% Номера страниц сверху и по центру
\def\footfont{\small}
\pagestyle{footcenter}
\chapterpagestyle{empty}

% Точка с запятой в качестве разделителя между номерами цитирований
% \setcitestyle{semicolon}

% Использовать полужирное начертание для векторов
\let\vec=\mathbf

% Включать подсекции в оглавление
\setcounter{tocdepth}{3}

\graphicspath{{pictures/}}


\begin{document}
\institution{Московский физико-технический институт (государственный университет) \\
    Факультет биологической и медицинской физики \\
    Кафедра кафедра молекулярной и трансляционной медицины}

% Имя лица, допускающего к защите (зав. кафедрой)
\apname{Лазарев В.Н.}

\title{Выпускная квалификационная работа\\[-14pt]на соискание степени\\МАГИСТРА}

\topic{Протеогеномный анализ штамма\\\ti{Mycobacterium tuberculosis W-148}}

% Автор
\author       {Смоляков А.В.} % ФИО
\group        {1114} % Группа
\coursenum    {010900} % Номер направления
\course       {Прикладные математика и физика}
\masterprognum{010982} % Номер магистерской программы
\masterprog   {Физико-химическая биология и биотехнология}

% Научный руководитель
\sa      {Шитиков Е.А.}
\sastatus{к. б. н.}

% Город и год
\city{Работа выполнена в ФГБУ ФНКЦ ФХМ ФМБА России\\Москва}
\date{\number\year}

\maketitle

\tableofcontents
\section{Список сокращений}
\noindent
\fl{GSSP --- Genome Search Specific Peptides} \\
\fl{PSM --- Peptide Spectrum Match}\\
МБТ --- микобактерии туберкулёзного комплекса

\newpage

\section{Введение}
\ti{Mycobacterium tuberculosis} является возбудителем тяжелой болезни - туберкулеза, наносящей большой вред здоровью. Особенно остро проблема стоит в развивающихся странах. Появление резистентных и более вирулентных штаммов усугубило ситуацию. Было сделано множество работ по исследованию протеома этого патогена \cite{jungblut1999comparative, mattow2006proteins}. Есть несколько работ, посвященных аннотации генома \ti{M.tubeculosis}. Полногеномное секвинирование штамма \ti{Mycobacterium tuberculosis H37Rv} было проведено в 1998 году, затем был отсекверирован штамм \ti{CDC1551} и несколько других \cite{cole1998erratum, fleischmann2002whole}. Точная аннотация белок-кодирующих генов является постоянно меняющимся и модернизирующимся техническим процессом. Это особено заметно в случае \ti{M.tubeculosis}. В работе Коула и его колег, было показано наличие 3924 ORF в геноме \ti{H37Rv} \cite{cole1998erratum}. При повторной аннотации, проведенной теми же авторами, число генов стало 3995 \cite{camus2002re}. В марте 2011 база \fl{TubercuList} содеражала 4012 аннотированых белок-кодирующих генов. де Соуза с коллегами провели сравнение двух различных аннотаций для \ti{H37Rv}, сделанных группами \fl{Sanger, TIGR}, и обнаружили, что 50\% генов имеют различные стартовые сайты \cite{de2008high}. В этой же работе, используя протеомные данные 449 культур, авторы смогли исправить аннотацию 24 генов. Так же, возможность существования CDSs, которые ещё не были аннотированы в \ti{H37Rv}, были предложены в работе Лью и коллег \cite{lew2011tuberculist}.

\newpage

\section{Литературный обзор}
тест~\cite{De_La_Cochetiere_2008, Lofmark_2006, Tanaka_2009}

\newpage

\section{Материалы и методы}

\subsection{Получение бактерий}

\subsection{Проведение масс-спектрометрического эксперимента}

\subsection{Контроль качества}


\subsection{Протеогеномика}
\subsubsection{Создание поисковых баз}
В работе использовалось 2 типа баз: белковая и геномная. Белковая база - аннотированные последовательности, для данного штамма. Геноманя - база, полученная в результате транслирования генома в шести рамках.
Белковые базы для \ti{M.tuberculosis W-148} и \ti{M.tuberculosis H37Rv} были составлены из аннотированных белков штаммов (NCBI Reference Sequence: NZ\_CP012090.1, 4244 аминокислотных последовательностей для \ti{W-148} и ).
Геномные базы были получены в результате 6 рамочного транслирования от стоп- до стоп-каднона геномов штаммов \ti{M.tuberculosis W-148} и \ti{M.tuberculosis H37Rv}, используя программу \fl{Artemis} версия 16.0.0 \cite{rutherford2000artemis}. При транслировании использовалась 11 трансляционная таблица \fl{NCBI}. Минимальная длинна рамки была установлена в 100 нуклеиновых кислот.
К каждой базе были добавлены последовательности 26 контаминантных белков (кератины, альбумины, трипин).

\subsubsection{Идентификация новых белков}

\subsubsection{Уточнение N-конца}

\subsubsection{Анализ SAP}

\subsection{Идентификация пептидов и белков}
% написать общий план идентифкации про 2 машины + мердж результатов
Данные полученные в результате LC-MS/MS эксперимента (Raw формат) были сконвертированы в пик-лист (MGF формат), используя \fl{ProteoWizard msconvert} \cite{chambers2012cross}. Идентификация проходила против двух белковых и двух геномных баз с использованием \fl{Mascot Search Engine version 2.5.1} \cite{cottrell1999probability}. Параметры поиска были следующими: триптические пептиды, не более двух пропущенных сайтов трипсинолиза, ошибка массы прекурсера 20 ppm, ошибка массы фрагментов 0.5 Да, заряды прекурсера 2+, 3+, 4+. \fl{Oxidation(M), Carbamidomethylation(C) and Deamidated(NQ)} были устанолвены как возможнные модификации пептидов. Для подсчета FDR и порогового скоринга использовался поиск против decoy-базы, полученной в результате реверса исходной базы. FDR был выбран на уровне 5\%.
Пептид считался идентифицированным, если его скор выше порогового скоринга и ранг равен еденице. Белок считался идентифицированным, если для него нашлось два и более уникальых пептидов.

\subsection{Количественный анализ}
\subsubsection{Анализ в MaxQuant}
\subsubsection{Анализ в Progenesis LC-MS}
\subsubsection{Сравнение результатов программ}



\newpage

\section{Результаты и обсуждение}

\subsection{Подготовка и проведение масс-спектрометрических измерений}
Тут что-нибудь про бактерий напиать.
Были отсняты масс-спектры 30 культур. Из них 6 были не фракционированы, остальные были разделены на 6 фракций. Разделение на фракции позволило идентифицировать пептиды, которые не были бы идентифицированы при обычном поиске, из-за того, что на измерения MS/MS спектра были бы отобраны более высоко представленные пептиды. Количество отснятых спектров для каждой культуры представлены в таблице.

\subsection{Подготовка баз}
Скаченная с NCBI аннотация содержала 4103 аннотированных белок-кодирующих последовательностей. После добавления контаминант и decoy-последовательностей получилась белковая база, состоящая из 8258 последовательностей. После транслирования генома в 6 рамках и исключения коротких последовательностей получилось 74488 белок-кодирующих последовательностей. В результате добавления контаминант и decoy-последовательностей, получилась геномная база, состоящая из 149028 последовательностей. Исключение коротких последовательностей из геномной базы позволило ускорить поиск; снизить пороговые значения идентификации, тем самым повысив чувствительность подхода. Минимальная длинна аннотированного белка \ti{W-148} составляет 84 аминокислоты, таким образом, исключение рамок длинной меньше чем 33 аминокислоты не должно привести к потерям при идентификации.

\subsection{Идентификация белков и пептидов}
Поиск проходил при помощи поисковых маших Mascot и X!Tandem. Объединение результатов поиска и перерасчет FDR был произведен в Scaffold. Против белковой базы было идентифицировано 32054 пептида (1041059 psm), против геномной базы 36502 уникальных пептида (1131085 psm). Пересечение идентифицированных пептидов представлено на рисунке . Часть пептидов идентифицирована против белковой базы и не идентифицирована против более полной полной геномной базы. Это связано с различными пороговыми скорингами при поиске против баз разных размеров. 
После вычитания результатов поиска против белковой базы из результатов поиска против геномной базы получилось 6015 GSSP (Genome search specific peptides).  После исключения пептидов представленных в аннотированном геноме и идентифицированных только против геномной базы, осталось 1397 GSSP. Наличие таких пептидов, идентифицируемых только против геномной базы и представленных в аннотации, связано с пересчетом FDR: при отдельном рассмотрении результатов идентификации каждой поисковой машины таких эффектов не возникает.  После исключения GSSP, представленных только в одной культуре, осталось 425 пептидов. Результаты интерпретации GSSP при таких фильтрах: 16 новый ORF, и 304 гена с корректированным положением старта.  
Пептидов, интерпретируемых как пептиды перед аннотированным стартом примерно в 8 раз больше, чем пептидов, относящихся к новым генам. Такой разброс может быть связан с тем, что пептидам относящимся к корректировке рамки проще пройти порог FDR. В самом деле, при пептидном и белковом FDR в 1\% и, примерно, 1000000 psm при размере базы в 4000 аминокислотных последовательностей, 10000 пептидов будут ложно-положительно идентифицированы. Если брать критерий 2 и более пептидов для идентификации белка, то такого количества пептидов будет достаточно для идентификации 5000 белков, если не учитывать белковой FDR. С учетом белкового FDR количество ложно-положительно идентифицированных белков должно быть не боле 40. Для этого достаточно 80 пептидов. Таким образом, в экспериментах с большим количеством исходных данных, белковый FDR становится более жестким критерием, чем пептидный. Соответственно, GSSP относящимся к корректировке рамки проще пройти белковый FDR, так как в этой рамке так же присутствуют пептиды из аннотированной части последовательности. В случае нового гена в "прохождение"\ белкового FDR участвуют только GSSP пептиды.

Для подтверждения результатов были применены дополнительные критерии. После удаления PSM, ошибка идентификации которых составляет более трех стандартных отклонений в предположении о нормальном распределении ошибки идентификации всех PSM в интервале +/- 5 минут, остался 331 уникальный GSSP. Следует отметить, что на этом шаге отсекались пептиды, ошибка идентификации которых меньше, чем реальная точность приборов. Поэтому данный критерий отнесен к дополнительным. Затем была проведена фильтрация по времени выхода. После исключения 10\% наиболее отклоняющихся пептидов, остался 147 уникальный GSSP. 

Все GSSP были проверены на точное вхождение в базу NCBInr. Среди белков, в которых нашлись GSSP, не было найдено  таких, которые бы относились к организмам, которые ранее снимались на используемом масс-спектрометре. Таким образом можно исключить остаточную контаминацию на приборе и пробоподготовке. Так же были исключены 8 пептидов, которые присутствуют в аннотированных последовательность \ti{W-148} с учетом одной замены.

\subsection{Интерпретация новых событий}
\subsubsection{Новые ORF}
Было идентифицировано 16 новых ORF, в которых присутствует 2 и более уникальных GSSP. Из шестнадцати ORF пять пересекаются с аннотированными псевдогенами, одиннадцать лежат на комплиментарной цепи участков с аннотированными генами. У пяти из шестнадцати есть гомолог в \ti{H37Rv}. Гены всех \ti{M.tuberculosis} плотно расположены, и межгенные области либо отсутствуют, либо их длинна намного меньше длины гена, либо, если межгенник большой, в нем находится псевдолен. Поэтому не найдено новых генов, которые не относились бы к псевдогенам и лежали в межгенном пространстве.

Следует отметить, что причины по которым ген становится "псевдоегном"\ с точки зрения биологии и системы аннотации NCBI различны. Так, наиболее частыми причинами из-за которых участок генома аннотируется как псевдоген являются: фреймшифт (потеря или вставка не кратного трем числа нуклеотидов, в результате чего нарушается белковая последовательность), неполный ген (присутвствует только часть гена, в сравнении с гамологами), стоп-кодон по середине последовательности, низкое качество сборки (например, если ген находится на стыке контигов). Для всех идентифицированных псевдогенов была найдена "техническая"\ и "биологическая"\ причина, из-за которой они получили статус псевдогена.

В результате поиска против базы NCBInr при помощи алгоритма blust для десяти из одинналцати ORF, лежащих на комплиментарной цепи к аннотированным генам, были найдены гомологи. Эти гомологи были аннотированы как “гипотетический/предсказанный/непроверенный белок” другого штамма \ti{M.tuberculosis}. Косвенно результаты идентификации подтверждает тот факт, что все новые ORF лежат на комплиментарной цепи, а не в другом фрейме аннотированной. В самом деле, с пространственной точки зрения предположение, что транскрипт снимается с комплиментарной цепи выглядит более вероятным, чем предположение, что с одного транскрипта идет трансляция в двух фреймах двух разных аминокислотных последовательностей.

После применения дополнительных критериев фильтрации GSSP осталось шесть ORF с псевдогенами и один ORF на комплиментарной цепи.

\subsubsection{Корректировка положения аннотированного старта}
Всего 304 рамки содержат аннотированный ген и GSSP пептид. Из 304 308 содержат два и более GSSP. Эти рамки сравниили с гомологичными генами в \ti{H37Rv}. Из 38 36 совпадают с точностью до SAP, 2 рамки длинней у \ti{W-148}, чем у \ti{H37Rv}. Для этих 38 рамок были проверены пептиды, пересекающиеся аннотированный старт. Для 17 из 38 были найдены пептиды, содержащие в себе аннотированный старт (overlap-пептиды), для 7 были найдены стартовые пептиды из аннотации, для 3 были найдены как стартовые, так и аннотированные пептиды.




\newpage



































\section{Выводы}
В результате протеогеномного анализа штамма \ti{Mycobacterium tuberculosis W-148} были идентифицированы 16 новых генов и у 249 (24) открытых рамок считывания были скорректированы старты трансляции

Проведенная реаннотация генома штамма \fl{Mycobacterium tuberculosis W-148} была подтверждена несколькими биоинформатическими подходами

\newpage

\bibliographystyle{gost705}
\bibliography{bibliography}
\end{document}

% рекомендуется оформлять текст шрифтом Times New Roman от 12 до 14 pt
% межстрочный интервал 1.5
% выравнивание в абзацах по ширине
% поля на странице: левое – 30 мм, остальные 20 мм.



