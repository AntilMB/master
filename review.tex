\section{Обзор литературы}

\subsection{Mycobacterium tuberculosis}

\subsection{Применение масс-спекртрометрии в протеомике}

\subsection{Orbitrap}

\subsection{Протеогеномика}
Для получения протеомных данных обычно используют метод <<\fl{shotgun proteomics}>> - комбинация жидкостной хроматографии и тандемной масс-спектрометрии \cite{bantscheff2012quantitative}. Одним из ключевых шагов в протеомике является идентификация пептидов на основе полученых MS/MS спектров. В отличае от геномных технологий, вроде ДНК или РНК секвенирования, где происходит непосредственное секвенирование исходной последовательности, в протеомике, как правило, пептиды идентифицируются за счет сопоставления экспериментальных MS/MS спектров и теоретических спектров всех пептидов, представленных в базе, против которой осуществляется поиск \cite{nesvizhskii2010survey}.
При таком поиске используются следующие исходные предположения: 
\begin{inparaenum}
    \item все белок-кодирующие последовательности генома точно известны и аннотированы
    \item все эти последовательности включены в базу, против которой осуществляется поиск
\end{inparaenum}.
Весь последующий анализ, включая идентификацию, количественный анализ и прочий статистический анализ, основывается на этих предположениях \cite{nesvizhskii2005interpretation}.

Проблема такого подхода заключается в том, что не все пептиды представлены в текущей поисковой базе или какой-либо другой. Пептиды могут содержать мутации, находиться в новых генах, перед неверно аннотированым стартом или в альтернативных сплайсформах. 
Один из способов идентификации пептидов с мутациями заключается в масс-тег подходе. При этом подходе происходит идентификация коротких участков пептида, после чего осуществляется поиск в более широком диапазоне масс прекурсеров \cite{dasari2010tagrecon}. 

Более общим подходом являтеся протеогеномика. Термин впервые был использован в 2004 и изначально использовался в исследовании, где протеомные данные использовались для улучшения качества аннотации \cite{jaffe2004proteogenomic}. С тех пор этот термин используется в более общем смысле. В протеогеномном подходе, пептиды идентифицируются за счет идентификации MS/MS спектрво против специальной базы, включающей в себя последовательности новых, предсказанных белков и различные варианты последовательности белка. Такие базы получаются за счет использования геномной и транскриптомной информации. Таким образом, протеогеномика позволяет не только подтвердить текущую аннотацию, но так же уточнить её \cite{nesvizhskii2014proteogenomics}. 

\subsubsection{Типы пептидов, идентифицируемых при протеогеномном исследовании}
Пептиды, идентифицируемые при протеогеномном поиске, соответствуют различным участкам генома. Такие пептиды можно разделить на межгенные (находся между аннотироваными генами) и внутрегенные (находятся полностью или частично в областях, где содержится аннотированный ген). Внутрегенные можно разделить на 
\begin{inparaenum}
    \item находящиеся в белоккодирующих генах 
    \item находящиеся в длинных некодирующих РНК
    \item находящиеся в псевдогенах
\end{inparaenum} \cite{harrow2012gencode}. Большинство пептидов, идентифицируемых в протеогеномике, уже известны и относятся к аннотированным генам. В эукариотах (в которых присутствует интроно-экзоная структура) большинство пептидов относятся к экзомам, и, как правило, меньше 20\% относятся к экзон-экзон участкам. Новые пептиды, не идентифицируемые против какой-либо базы, могут находится в неаанотированных участках генома, быть результатами одно-аминокислотной замены (SAP), находится в нетранслируемых регионах (3' или 5' UTR) или интронах, явлся результатам альтернативного сплайсинга \cite{nesvizhskii2014proteogenomics}.

\subsubsection{Подходы к созданию поисковых баз}
Идентификация пептидов против против кастомных баз являтеся ключевым шагом в протеогеномике. Обычно база состоит из известных аннотированных последовательностей и предсказанных последовательностей. При протеогеномном поиске следует внимательно относиться к размеру базы: увелчение размера влечет за собой увлечение времени поиска и FDR. Оптимальный выбор зависит от того, что требуется в эксперименте: точность или чувствительность \cite{nesvizhskii2014proteogenomics}.

\textbf{Транслирование в шести рамках генома} - такая база может получена в результате транслирования в шести рамках генома \cite{baerenfaller2008genome}. Недостатком такого подхода является гигантский размер итоговой базы (в основном состоящей из последовательностей несуществующих белков) и невозможности поиска экзом-экзом пептидов, в случае эукариот. Например транслированный таким образом геном человека приводит к базе в ~3.2 гигабазы белковых последовательностей, что в 70 раз больше, чем референс в 45 мегабаз \cite{khatun2013whole}. Для уменьшения размера базы могут применены различные вычислительные методы: выбор последовательностей, имеющих гомологии с уже известными белками; использование методов предсказания кодирующего потенциала; исключение слишком коротких последовательностей (например, меньших, чем 30 аминокислот) \cite{blakeley2012addressing}. 

\textbf{Ab initio предсказание генов} 

\textbf{Expressed sequence tag (EST) data}

\textbf{Аннотированные РНК-транскрипты} Белковые последовательности могут быть получены в результате шестирамочного транслирования аннотированых РНК-транскриптов, например Enselbl или RefSeq. Это позволяет идентифицировать альтернативные сайты инициации трансляции. База GENCODE содержит 84408 мРНК аннотированых белков. В результате транслирования такой базы получается белковая база в 200 мегабазы, что всего в 4.5 раза больше референса \cite{khatun2013whole}. Так же такие базы могут содержать последовательность, аннотированые как псевдогены или длинные некодирующие РНК \cite{derrien2012gencode}.

\textbf{RNA-seq данные}

\textbf{Различные вариации последовательностей} Белковые последовательности в рефернсной базе могут быть дополнены последовательностями, являющими вариациями референсных последовательностей (как правило, это одно аминокислотные полиморфизмы, делеции и инсерции). Для каждой вариации, берется большая область вокруг вариации и добавляется в базу, как независимая последовательность. Одно аминокислотные замены можно скачать из различных баз данных: \fl{NCBI dbSNP, Online Mendelian Inheritance, Protein Mutant Database} \cite{li2011bioinformatics}.  

\textbf{Прочие специализированные базы}

\subsection{Влияние размера базы}


\subsubsection{Поиск новых генов и корректировка рамок}
\subsubsection{Причины приводящие к неточности аннотации}

\newpage
