\section{Обзор литературы}

\subsection{Mycobacterium tuberculosis}

\subsection{Применение масс-спекртрометрии в протеомике}

\subsection{Orbitrap}

\subsection{Протеогеномика}
Для получения протеомных данных обычно используют метод <<\fl{shotgun proteomics’}>> - комбинация жидкостной хроматографии и тандемной масс-спектрометрии \cite{bantscheff2012quantitative}. Одним из ключевых шагов в протеомике является идентификация пептидов на основе поулченых MS/MS спектров. В отличае от геномных технологий, вроде ДНК или РНК секвенирования, где происхоид непосредственное секвенирование исходной последовательности, в протеомике, как правило, пептиды идентифицируются за счет сопоставления экспериментальных MS/MS спектров и теоретических спектров всех пептидов, представленных в базе, против которой осуществляется поиск \cite{nesvizhskii2010survey}.
Используются следующие исходные предположения: 
\begin{inparaenum}
    \item все белок-кодирующие последовательности генома точно известны и аннотированы
    \item все эти последовательности включены в базу, против которой осуществляется поиск
\end{inparaenum}.
Весь последующий анализ, включая идентификацию, количественный анализ и прочий статистический анализ, основывается на этих предположениях \cite{nesvizhskii2005interpretation}.

Проблема такого подхода заключается в том, что не все пептиды представлены в текущей поисковой базе или какой-либо другой. Пептиды могут содержать мутации, находиться в новых генах, перед неверно аннотированым стартом или в альтернативных сплайсформах. 
Один из способов идентификации пептидов с мутациями заключается в масс-тег подходе. При этом подходе происходит идентификация коротких участков пептида, после чего осуществляется поиск в более широком диапазоне масс прекурсеров \cite{dasari2010tagrecon}. 

Более общим подходов являтеся протеогеномика. Термин впервые был использован в 2004 и изначально использовался в исследовании, где протеомные данные использовались для улучшения качества аннотации \cite{jaffe2004proteogenomic}. С тех пор этот термин используется в более общем смысле. В протеогеномном подходе, пептиды идентифицируются за счет идентификации MS/MS спектрво против специальной базы, включающей в себя последовательности новых, предсказанных белков и различные варианты последовательности белка. Такие базы получаются за счет использования геномной и транскриптомной информации. Таким образом, протеогеномика позволяет не только подтвердить текущую аннотацию, но так же уточнить её \cite{nesvizhskii2014proteogenomics}. 

\subsubsection{Подходы к созданию баз}
\subsubsection{Поиск новых генов и корректировка рамок}
\subsubsection{Причины приводящие к неточности аннотации}

\newpage
