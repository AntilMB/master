\section{Обзор литературы}

\subsection{Mycobacterium tuberculosis}
\ti{Mycobacterium tuberculosis} – возбудитель туберкулеза (ТБ), заболевания известного еще с древности и являющегося одной из главных причин смертности во всем мире. По данным Всемирной организации здравоохранения  за 2014 год выявлено 9.6 миллионов новых и 1.5 миллиона смертельных случаев ТБ (WHO, 2015). При этом Россия является 1 из 22 стран с наивысшем бременем туберкулеза, на долю которых приходиться 80\% всех случаев ТБ. 

Mycobacterium tuberculosis являются грамположительными палочками, длиной 1-10 мкм и диаметром 0.2-0.6 мкм. Морфологически выделяют, как прямые, так и слегка изогнутые формы.

\subsubsection{Особенности генома}
Последовательность генома M. tuberculosis штамма H37Rv была полностью расшифрована в Сенгеровском институте в 1998 году (Cole et al., 1998) (Рисунок 1). Это был третий опубликованный бактериальный геном после Haemophilus influenzae (Fleischmann et al., 1995) и Mycoplasma genitalium (Fraser et al., 1995).

Следует отметить, что ключевые особенности организации генома одинаковы для всех штаммов патогена. Геномы представлены кольцевой молекулой ДНК протяженностью около 4400 тысяч пар оснований (т.п.о.) и характеризуются высоким содержанием GC пар (\~ 65.5\%). При этом существует несколько регионов, отличающихся по GC составу.

Углубленный анализ генома M. tuberculosis штамма H37Rv выявил около 4000 генов, кодирующих белки. При этом следует отметить, что альтернативный старт трансляции GTG встретился в 35\% случаев, что существенно чаще, чем 14\% и 9\% в геномах Bacillus subtilis и Escherichia coli, соответственно. Был найден один набор рибосомных генов и 45 транспортных РНК.

\subsubsection{Генетическое семейство Beijing}
Впервые представители генотипа Beijing были обнаружены в 90х годах ХХ века, в двух независимых исследованиях, проведенных группами исследователей из Голландии и Америки (van Soolingen et al., 1995; Bifani et al., 1996). В ходе IS6110 RFLP анализа и сполиготипирования коллекции изолятов M. tuberculosis, полученных от больных ТБ в 1992-1994 годах в Китайской Народной Республике и Монголии, van Soolingen с соавт. выявили доминирующий генотип. При этом наиболее часто представители генотипа встречались в окрестностях Пекина (англ. Beijing), отчего и получили свое название (van Soolingen et al., 1995). Параллельно этому исследованию, Bifani с соавт. из Научно-исследовательского института общественного здравоохранения США (Public Health Research Institute, NY, США), в 1996 году методами молекулярно-генетической эпидемиологии описали вспышку лекарственно-устойчивого туберкулеза, произошедшую в Нью-Йорке в начале 1990х. Выявленные штаммы характеризовались крайне схожими паттернами IS6110 профилей и были названы «W» (Bifani et al., 1996). В дальнейшем эти названия были объединены в W/Beijing или просто Beijing (Kurepina et al., 1998; Van Soolingen, 2001). При этом название как нельзя лучше отражает реальное место зарождения генотипа. Его представители наиболее часто встречаются в Восточной Азии и, по мнению Мокроусова с соавт., генотип Beijing возник в Северном Китае более 1,000 лет назад. Дальнейшее его распространение было связано с миграционными потоками: со средневековых времен в Россию, совсем недавно в ЮАР (с XVII века) и в Австралию (в XIX веке) (Mokrousov et al., 2008). В свою очередь, согласно Merker с соавт., генотип в целом возник более 6,000 лет назад в географической зоне, включающей в себя Северо-Восток Китая, Корею и Японию (Merker et al., 2015).

Согласно международной базе данных сполиготипирования SpolDB4, штаммы Beijing присутствуют в наибольшем количестве стран на глобальном уровне (13\% от мирового количества изолятов), являясь по этому показателю уникальным генотипом (Brudey et al., 2006). Здесь также следует отметить ассоциацию представителей генотипа с многочисленными вспышками заболеваний во всем мире, многие из которых были лекарственно устойчивые (Frieden et al., 1996; Agerton et al., 1999; Caminero et al., 2001; Affolabi et al., 2009). В структуре популяции возбудителя туберкулеза в России доля штаммов Beijing составляет от 50\% до 80\% (Mokrousov et al., 2003), причем, крайне выражена ассоциация штаммов с лекарственной устойчивостью (Casali et al., 2014). Исходя из сказанного выше, предполагается, что у штаммов данной эволюционной линии, возможно, развились уникальные свойства, которые позволили им распространиться по всему миру (клональная экспансия). По мнению многих авторов, этими свойствами являются: 1) способность «ускользать» от БЦЖ-вакцинирования 2) способность штаммов относительно быстро приобретать устойчивость к противотуберкулезным препаратам.


\subsection{Применение масс-спектрометрии в протеомике}
Стандартный эксперимент по исследованию белков с использованием масс-спектрометра состоит из пяти основных этапов. На первом шаге белки из клеточного лизата или ткани извлекаются и очищаются. Как правило, за этим следует разделение полученной смеси гель-электрофорезом. Следующим этапом является фрагментирование белка на пептиды. Обычно это происходит при помощи фермента трипсина. На третьем этапе пептиды, предварительно разделенные жидкостной хроматографией на одну или несколько фракций, подвергаются ионизации и поступают в масс анализатор. После MS-анализа пептиды могут подвергнуться повторной фрагментации. Новые ионы так же подвергаются анализу. Это пятый этап (MS/MS или тандемная масс-спектрометрия). МС-основанная протеомика зарекомендовала себя как незаменимая технология для определения закодированной в геноме информации. На сегодняшний момент белковый анализ (первичная структура, посттранскрипционные модификации или белок-белковые взаимодействия) при помощи МС является наиболее успешным при работе с небольшим (по сравнению с другими методами белкового анализа) количеством белка, выделенным из различных образцов. Системный анализ большого количества генов, экспрессированных в клетки, является основной целью протеомики; эта область сейчас быстро развивается, в основном, благодаря разработке новых экспериментальных подходов.

\subsection{Orbitrap}

\subsection{Методы протеогеномики}
Для получения протеомных данных обычно используют метод <<\fl{shotgun proteomics}>> - комбинация жидкостной хроматографии и тандемной масс-спектрометрии \cite{bantscheff2012quantitative}. Одним из ключевых шагов в протеомике является идентификация пептидов на основе полученных MS/MS спектров. В отличие от геномных технологий, вроде ДНК или РНК секвенирования, где происходит непосредственное секвенирование исходной последовательности, в протеомике, как правило, пептиды идентифицируются за счет сопоставления экспериментальных MS/MS спектров и теоретических спектров всех пептидов, представленных в базе, против которой осуществляется поиск \cite{nesvizhskii2010survey}.
При таком поиске используются следующие исходные предположения: 
\begin{inparaenum}
    \item все белок-кодирующие последовательности генома точно известны и аннотированы
    \item все эти последовательности включены в базу, против которой осуществляется поиск
\end{inparaenum}.
Весь последующий анализ, включая идентификацию, количественный анализ и прочий статистический анализ, основывается на этих предположениях \cite{nesvizhskii2005interpretation}.

Проблема такого подхода заключается в том, что не все пептиды представлены в текущей поисковой базе или какой-либо другой. Пептиды могут содержать мутации, находиться в новых генах, перед неверно аннотированным стартом или в альтернативных сплайсформах. 
Один из способов идентификации пептидов с мутациями заключается в масс-тег подходе. При этом подходе происходит идентификация коротких участков пептида, после чего осуществляется поиск в более широком диапазоне масс прекурсеров \cite{dasari2010tagrecon}. 

Более общим подходом является протеогеномика. Термин впервые был использован в 2004 и изначально использовался в исследовании, где протеомные данные использовались для улучшения качества аннотации \cite{jaffe2004proteogenomic}. С тех пор этот термин используется в более общем смысле. В протеогеномном подходе, пептиды идентифицируются за счет идентификации MS/MS спектров против специальной базы, включающей в себя последовательности новых, предсказанных белков и различные варианты последовательности белка. Такие базы получаются за счет использования геномной и транскриптомной информации. Таким образом, протеогеномика позволяет не только подтвердить текущую аннотацию, но так же уточнить её \cite{nesvizhskii2014proteogenomics}. 

\subsubsection{Типы пептидов, идентифицируемых при протеогеномном исследовании}
Пептиды, идентифицируемые при протеогеномном поиске, соответствуют различным участкам генома. Такие пептиды можно разделить на межгенные (находятся между аннотированными генами) и внутрегенные (находятся полностью или частично в областях, где содержится аннотированный ген). Внутрегенные можно разделить на 
\begin{inparaenum}
    \item находящиеся в белок-кодирующих генах 
    \item находящиеся в длинных некодирующих РНК
    \item находящиеся в псевдогенах
\end{inparaenum} \cite{harrow2012gencode}. Большинство пептидов, идентифицируемых в протеогеномике, уже известны и относятся к аннотированным генам. В эукариотах (в которых присутствует интроно-экзоная структура) большинство пептидов относятся к экзомам, и, как правило, меньше 20\% относятся к экзон-экзон участкам. Новые пептиды, не идентифицируемые против какой-либо базы, могут находится в неаанотированных участках генома, быть результатами одно-аминокислотной замены (SAP), находится в нетранслируемых регионах (3' или 5' UTR) или интронах, является результатам альтернативного сплайсинга \cite{nesvizhskii2014proteogenomics}.

\subsubsection{Подходы к созданию поисковых баз}
Идентификация пептидов против кастомных баз является ключевым шагом в протеогеномике. Обычно база состоит из известных аннотированных последовательностей и предсказанных последовательностей. При протеогеномном поиске следует внимательно относиться к размеру базы: увеличение размера влечет за собой увлечение времени поиска и FDR. Оптимальный выбор зависит от того, что требуется в эксперименте: точность или чувствительность \cite{nesvizhskii2014proteogenomics}.

\textbf{Транслирование в шести рамках генома} - такая база может получена в результате транслирования в шести рамках генома \cite{baerenfaller2008genome}. Недостатком такого подхода является гигантский размер итоговой базы (в основном состоящей из последовательностей несуществующих белков) и невозможности поиска экзом-экзом пептидов, в случае эукариот. Например транслированный таким образом геном человека приводит к базе в ~3.2 гигабазы белковых последовательностей, что в 70 раз больше, чем референс в 45 мегабаз \cite{khatun2013whole}. Для уменьшения размера базы могут применены различные вычислительные методы: выбор последовательностей, имеющих гомологии с уже известными белками; использование методов предсказания кодирующего потенциала; исключение слишком коротких последовательностей (например, меньших, чем 30 аминокислот) \cite{blakeley2012addressing}. 

\textbf{Ab initio предсказание генов} 

\textbf{Expressed sequence tag (EST) data}

\textbf{Аннотированные РНК-транскрипты} Белковые последовательности могут быть получены в результате шестирамочного транслирования аннотированных РНК-транскриптов, например Enselbl или RefSeq. Это позволяет идентифицировать альтернативные сайты инициации трансляции. База GENCODE содержит 84408 мРНК аннотированных белков. В результате транслирования такой базы получается белковая база в 200 мегабазы, что всего в 4.5 раза больше референса \cite{khatun2013whole}. Так же такие базы могут содержать последовательность, аннотированные как псевдогены или длинные некодирующие РНК \cite{derrien2012gencode}.

\textbf{RNA-seq данные}

\textbf{Различные вариации последовательностей} Белковые последовательности в рефернсной базе могут быть дополнены последовательностями, являющими вариациями референсных последовательностей (как правило, это одноаминокислотные полиморфизмы, делеции и инсерции). Для каждой вариации, берется большая область вокруг вариации и добавляется в базу, как независимая последовательность. Одно аминокислотные замены можно скачать из различных баз данных: \fl{NCBI dbSNP, Online Mendelian Inheritance, Protein Mutant Database} \cite{li2011bioinformatics}.  

\textbf{Прочие специализированные базы}

\subsubsection{Влияние размера базы}
Возможность идентифицировать спектры пептидов, полученные в результате MS/MS экспериментов, используя поисковые базы данных, зависит от многих факторов. Во-первых, пептид должен присутствовать в поисковой базе. Однако, чем больше сравнивается теоретических спектров с экспериментальным, тем больше вероятность того, что лучший результат будет у неверного теоретического спектра, и тем труднее различать верные и неверные идентификации \cite{nesvizhskii2010survey}. В результате, поиск против большой базы может дать несколько новых идентификаций белков или пептидов, но при этом общее количество идентификаций будет меньше, в сравнении с поиском против референсного сиквенса \cite{blakeley2012addressing, krug2013deep}. Так же увеличение базы приводит к увеличению машинного времени, необходимого для этого поиска. Таким образом, одним из ключевых моментов в протеогеномике является поиск баланса между размером базы и её содержимым.

\subsubsection{Способы увелечение чувствительности идентификации пептидов}
Подходы, используемые в протеомике для увелечения числа идентификаций, включают в себя идентификацию с помощью нескольких поисковых машин одного и того же набора данных \cite{shteynberg2013combining}. После поиска происходит изменение скорингов идентификации пептидов за счет комбинации нескольких источников информации, используя подходы машинного обучения \cite{nesvizhskii2010survey}. Одной из дополнительных стратегий для сокращения пространства поиска является фракционирование смеси пептидов, проводимое до \fl{LS-MS/MS} анализа. Фракционирование может проходить за счет определенных физико-химических свойств пептида или свойств последовательности пептида. Примером фракционирования может служить изоэлектрическая фокусировка. Спектры, полученные от фракции с определенной изоэлектрической точкой, можно искать против пептидов с примерно такой же теоретически предсказанной изоэлектрической точкой \cite{branca2014hirief}. 

Чувствительность идентификации так же можно повысить за счет многоступенчатого анализа данных. На первом этапе можно проводить поиск против \textquotedblleftэталонной\textquotedblright\ базы, наиболее точно описывающей исследуемый организм и позволяющей идентифицировать большинство спектров. На втором этапе происходит поиск против расширенной базы для идентификации дополнительных спектров \cite{ning2010computational, helmy2012mass}. При таком подходе результаты первоначального поиска используются для уточнения второй базы, используемой при дальнейшем анализе. 

\subsubsection{Точность идентификации}
В протеогеномике, как и в протеомике, для предотвращения накопления ошибок при переходе с PSM на уровень уникальных пептидов, избыточные PSM должны быть свернуты в один PSM с наивысшим скорингом \cite{nesvizhskii2010survey}. Пептиды, идентифицированные в различных состояниях (например, двух- или трех-зарядные ионы; модифицированные и не модифицированные формы) так же должны быть объединены или обрабатываться вероятностно, с учетом удельных весов \cite{shteynberg2011iprophet}. При использовании многоэтапного поиска, когда результаты первой идентификации используются при подготовки базы для дальнейшей, более специфической идентификации, необходимо на каждом шагу добавлять соответствующие количество decoy-последовательностей в базу \cite{nesvizhskii2010survey}. Кроме контроля глобального FDR, необходимо следить за достоверностью каждой отдельной идентификации (например, при глобальном пептидном FDR в 5\% ошибка идентификации отдельных пептидов может превышать это значение). Вероятность идентификации белка или события (новый ген, альтернативный старт, сплайс форма, в случае протеогеномики, может быть рассчитана на основании вероятностей соотвествующих уникальных пептидов \cite{shteynberg2011iprophet, castellana2010proteogenomics}.

\subsubsection{Класс-специфический анализ идентификаций и FDR}


\subsubsection{Ложная неслучайная идентификация пептидов}
Ошибочная идентификация пептида происходит в одном из двух случаев: либо случайное совпадение с высоким скором MS/MS-спектра и несвязанного с ним пептидом из базы, либо в результате гомологичности последовательности пептида из базы и истинной последовательности пептида. Вне зависимости от типа используемой при поиске decoy-базы (например реверсивное или случайное прочтение исходных последовательностей), ложные идентификации второго типа, скорей всего, будут недооценены \cite{nesvizhskii2010survey}. Часто ложная идентификация происходит в результате химической модификации высоко представленного пептида, если в результате сдвига масс из-за модификации, масса пептида становится эквивалента массе некоторого другого пептида из базы \cite{nesvizhskii2006dynamic, abraham2013moving}. Для исключения таких идентификаций можно, например с использованием BLAST, проверить схожесть последовательности каждого нового идентифицированного пептида, с последовательностями всех пептидов, представленных в референсной базе, и исключить (или отдельо контролировать) все высоко гомологичные. Если нужно сохранить для дальнейшего анализа такие пептиды (например при поиске одно аминокислотных замен), нужно проверить, что наблюдаемая разница масс между новым и референсынм пептидом не совпадает с массой какой-либо распространенной химической или пост-трансляционной модификацией \cite{li2011bioinformatics}. Список наиболее частных модификаций, специфичных для исследуемого биологичесого объекта, можно получить, используя 'blind' поисковый алгоритм \cite{tsur2005identification}. Кроме того, замена лейцина на изолейцин и обратная не может быть идентифицирована с помощью масс-спектрометрии. Пептиды, содержащие такие замены, должны быть исключены из дальнейшего анализа.

\subsubsection{Интерпретация данных и новых событий}
В протеомике результатом идентификации является список идентифицированных белков или генов, а так же список идентифицированных пептидов, с определенным уровнем FDR. В протеогеномике к таким результатам, так же добавляются списки новых событий, вроде "новый ген", "новый кодирующий регион", "альтернативный старт" и так далее, с соответствующими пептидами, подтверждающие эти события. Пептиды с одинаковыми последовательностями, могут придти из различных областей генома (например, от паралогов, идентичных сайтов разных белков или из псевдогенов). Такие пептиды не могут быть доказательством экспресии с какого-то определенного участка генома \cite{nesvizhskii2005interpretation}. Кроме того, новые пептиды могут быть интерпретированы разными способами, например как новый транскрипт этого гена или как пептид из интрона или нетранслируемого региона того же гена. 

\subsubsection{Идентификация новых пептидов}
Одним из результатов протеогеномного анализа является список новых пептидов. Этот список зависит от выбора референсной базы и её версии. Как обсуждалось ранее, для многих организмов существуют несколько версий белковых баз, и эти базы отличаются объемом и качественном аннотации. Более того, эти базы регулярно обновляются, в результате в них добавляются и удаляются последовательности. Таким образом, в протеогеномным исследованиях, пептиды идентифицируемые против специфичных баз, должны быть проверены на вхождение в основанные базы, существующие для данного организма.

\subsection{Применение протеогеномики}

\subsubsection{Поиск новых белок-кодирующих регионов}
Возможность применение масс-спектрометрических протеомных данных для поиска новых белок-кодирующих регионов и подтверждения границ уже аннотированных, обсуждалась начиная с первых дней существования протеомики, как науки в её современном виде \cite{choudhary2001interrogating, andersen2001mass}. Такие результаты чаще всего достигаются в рез

\newpage
