\section{Обзор литературы}

\subsection{Mycobacterium tuberculosis}
Туберкулез, вызванный Mycobacterium tuberculosis, является одной из наиболее значимых проблем здравоохранения в мире. По оценке Всемирной организации здравоохранения (ВОЗ) Россия относится к 20 странам мира с наибольшим распространением данной инфекции. В настоящее время можно говорить о наметившейся тенденции к снижению заболеваемости, но, тем не менее, она продолжает оставаться на достаточно высоком уровне. Только за 2016 год в стране было выявлено около 100 тысяч вновь заболевших (WHO, 2016). Согласно современной таксономии \ti{M. tuberculosis} относится к царству Bacteria, типу Actinobacteria, классу Actinobacteridae, подклассу Actinomycetales, отряду Firmicutes, подотряду Corynebacterineae, семейству Mycobacteriaceae, роду Mycobacterium (Криг, Н. Определитель бактерий Берджи. В 2-х т. / Н. Криг, П. Снит, С. Уильямс, Э. Бок, Д. Хоулт, Р. Беркли, Д. Бун, Дж. Стейли, П. Стин: пер. с англ. − М. : МАКС Пресс, 2007). Большинство микобактерий относятся к сапрофитам, но есть небольшая группа патогенных бактерий. Род включает в себя свыше 100 видов бактерий, которые разделяют на две основные группы согласно их скорости роста: медленно и быстро растущие виды \cite{godreuil2007first}. Быстро растущие бактерии образуют видимые колонии на питательной среде при оптимальных условиях роста через семь дней. Данная группа включает в себя такие виды как \ti{Mycobacterium smegmatis, Mycobacterium fortuitum} и \ti{Mycobacterium abscessus}, представители которых обладают ограниченным патогенным эффектом и вызывают нетипичные клинические признаки у людей и животных. В противоположность этому, медленно растущая группа, включающая микобактерии туберкулезного комплекса (M. tuberculosis complex), \ti{Mycobacterium tuberculosis, Mycobacterium avium, Mycobacterium leprae} и \ti{Mycobacterium kansasii}, характеризуется большей патогенностью для человека и животных и вызывает хронические заболевания \cite{godreuil2007first}.

Бактерии \ti{M. tuberculosis} имеют вид прямой или слегка изогнутой палочки шириной от 0.2 до 0.6 мкм и длинной, варьирующей от 1 до 10 мкм \cite{godreuil2007first}. Это неподвижные, не спорообразующие, грамположительные аэробные или факультативно анаэробные бактерии. Микобактерии туберкулеза демонстрируют тропность к тканям легких. Они также являются факультативными внутриклеточными патогенами, инфицирующими макрофаги, где они размножаются и затем разносятся по различным частям тела/организма \cite{van2001molecular}. 


\subsubsection{Особенности генома и популяционная структура \ti{M. tuberculosis}}
Последовательность генома \ti{M. tuberculosis} штамма H37Rv была полностью расшифрована в Сенгеровском институте в 1998 году \cite{cole1998erratum} (Рисунок \ref{genome}). Это был третий опубликованный бактериальный геном после \ti{Haemophilus influenzae} \cite{fleischmann1995whole} и \ti{Mycoplasma genitalium} \cite{fraser1995minimal}.

\begin{figure}[h!]
    \begin{center}
        \includegraphics[width=0.9\linewidth]{genome.png}
    \end{center}
\caption[foo bar]{Круговая карта генома \ti{M. tuberculosis штамма} H37Rv.}
\label{genome}
\end{figure}

Следует отметить, что ключевые особенности организации генома одинаковы для всех штаммов патогена. Геномы представлены кольцевой молекулой ДНК протяженностью около 4400 тысяч пар оснований (т.п.о.) и характеризуются высоким содержанием GC пар (\~ 65.5\%). При этом существует несколько регионов, отличающихся по GC составу.

Углубленный анализ генома \ti{M. tuberculosis} штамма H37Rv выявил 3906 генов (аннотация RefSeq, \fl{NCBI Reference Sequence: NC\_000962.3}), кодирующих белки. При этом следует отметить, что альтернативный старт трансляции GTG встретился в 35\% случаев, что существенно чаще, чем 14\% и 9\% в геномах \ti{Bacillus subtilis} и \ti{Escherichia coli}, соответственно. Был найден один набор рибосомных генов и 45 транспортных РНК.

Первые результаты анализа геномных последовательностей \ti{M. tuberculosis} показали крайне малое разнообразие патогена на генетическом уровне по сравнению с другими видами бактерий. Дополнительно, методами популяционной генетики, была установлена крайняя степень клональности популяционной структуры \cite{sreevatsan1997restricted, ramaswamy1998molecular, hirsh2004stable}. Это обстоятельство в свое время наложило ограничение на использование метода мультилокусного типирования, столь широко применяемого для других бактериальных патогенов. Однако с развитием сравнительной геномики и усовершенствованием технологий полногеномного секвенирования на геномном уровне было найдено значительное количество вариаций, которые могут быть использованы для построения филогенетически надежных классификаций \cite{gagneux2007global}. Такими информативными маркерами для \ti{M. tuberculosis} послужили крупные делеции (large sequence polymorphisms, LSPs) и однонуклеотидные полиморфизмы (single nucleotide polymorphisms, SNPs). При этом важно подчеркнуть, что из-за достаточно малой вариации геномной последовательности патогена (штаммы в среднем отличаются друг от друга не более чем на 2,000 SNPs) вероятность возникновения независимой рекуррентной мутации крайне мала. Отсутствие горизонтального переноса генов дополнительно уменьшает возможность обратной мутации, а в случае с LSPs предотвращает появление геномных последовательностей, потерянных ранее. Основные работы, посвященные изучению популяционной структуры патогена, представлены на рисунке \ref{diser_pic_4}. Как видно из рисунка, в настоящее время выделяют 4 линии, относящиеся к \ti{M. tuberculosis}, и 2 линии, характеризующие \ti{M. africanum}.

\begin{figure}[h!]
    \begin{center}
        \includegraphics[width=0.9\linewidth]{diser_pic_4.png}
    \end{center}
\caption[foo bar]{Сравнение терминологии и молекулярных маркеров для определения основных линий внутри \ti{M. tuberculosis} и \ti{M. africanum}}
\label{diser_pic_4}
\end{figure}

\subsubsection{Генетическое семейство Beijing}
Впервые представители генотипа Beijing были обнаружены в 90х годах ХХ века, в двух независимых исследованиях, проведенных научными группами из Голландии и Америки \cite{van1995predominance, bifani1996origin}. В ходе IS6110 RFLP анализа и сполиготипирования коллекции изолятов \ti{M. tuberculosis}, полученных от больных ТБ в 1992-1994 годах в Китайской Народной Республике и Монголии, van Soolingen с соавт. выявили доминирующий генотип. При этом наиболее часто представители генотипа встречались в окрестностях Пекина (англ. Beijing), отчего и получили свое название \cite{van1995predominance}. Параллельно этому исследованию, Bifani с соавт. из Научно-исследовательского института общественного здравоохранения США (Public Health Research Institute, NY, США), в 1996 году методами молекулярно-генетической эпидемиологии описали вспышку лекарственно-устойчивого туберкулеза, произошедшую в Нью-Йорке в начале 1990х. Выявленные штаммы характеризовались крайне схожими паттернами IS6110 профилей и были названы «W» \cite{bifani1996origin}. В дальнейшем эти названия были объединены в W/Beijing или просто Beijing \cite{van2001molecular, kurepina1998characterization}. При этом название как нельзя лучше отражает реальное место зарождения генотипа. Его представители наиболее часто встречаются в Восточной Азии и, по мнению Мокроусова с соавт., генотип Beijing возник в Северном Китае более 1,000 лет назад. Дальнейшее его распространение было связано с миграционными потоками: со средневековых времен в Россию, совсем недавно в ЮАР (с XVII века) и в Австралию (в XIX веке) \cite{mokrousov2008molecular}. В свою очередь, согласно Merker с соавт., генотип в целом возник более 6,000 лет назад в географической зоне, включающей в себя Северо-Восток Китая, Корею и Японию \cite{merker2015evolutionary}.

Согласно международной базе данных сполиготипирования SpolDB4, штаммы Beijing присутствуют в наибольшем количестве стран на глобальном уровне (13\% от мирового количества изолятов), являясь по этому показателю уникальным генотипом \cite{brudey2006mycobacterium}. Здесь также следует отметить ассоциацию представителей генотипа с многочисленными вспышками заболеваний во всем мире, многие из которых были лекарственно устойчивые \cite{frieden1996multi, agerton1999spread, affolabi2009first, caminero2001epidemiological}. В структуре популяции возбудителя туберкулеза в России доля штаммов Beijing составляет от 50\% до 80\% \cite{mokrousov2003pcr}, причем, крайне выражена ассоциация штаммов с лекарственной устойчивостью \cite{casali2014evolution}. Исходя из сказанного выше, предполагается, что у штаммов данной эволюционной линии, возможно, развились уникальные свойства, которые позволили им распространиться по всему миру (клональная экспансия). По мнению многих авторов, этими свойствами являются: 1) способность «ускользать» от БЦЖ-вакцинирования 2) способность штаммов относительно быстро приобретать устойчивость к противотуберкулезным препаратам.

\subsubsection{Характеристика кластера Beijing В0/W148}
При описании генетического семейства Beijing необходимо учитывать, что это все-таки разнородная группа и даже внутри генотипа существуют более  «успешные» кластеры. Одним из таких кластеров является Beijing B0/W148. Впервые штаммы кластера были выявлены на рубеже ХХ и ХХI веков. В независимых исследованиях Нарвской, Курепиной и Portaels с использованием IS6110 RFLP типирования обнаружили группы кластеризующихся образцов, названные B0 (отечественная систематика) и W148 (иностранная систематика). Отличительной особенностью этих образцов было наличие двойной полосы (7.1 и 9.2 Kb) в верхней части профиля. В 2008 году штаммы Beijing B0/W148 с sensu stricto профилем и характерной двойной полосой были отнесены к Beijing B0/W148 \cite{mokrousov2008molecular}, а метод IS6110-RFLP типирования был признан «золотым стандартом» для выявления изолятов данной клональной группы. Дополнительными названиями кластера могут считаться CladeB \cite{casali2014evolution} и ECDC0002 (de Beer et al., 2014).

На сегодняшний день опубликовано достаточно много данных об ассоциации штаммов кластера с лекарственной устойчивостью, что является достаточным, для оценки степени опасности, исходящей от циркуляции изолятов B0/W148. Первая статья, описывающая российские штаммы \ti{M. tuberculosis}, выделенные в середине 1990-х, уже показала широкую распространенность МЛУ среди изолятов данного кластера \cite{marttila1998ser315thr}. Недавно изоляты B0/W148 с множественной лекарственной устойчивостью были выявлены в эпидемиологически значимой выборке пациентов с впервые выявленным ТБ в Ленинградской \cite{tratata123}, Тульской \cite{dubiley2010molecular}, Самарской и других областях. Следует отметить, что в исследовании более 1,000 штаммов из Самары 119 штаммов относилось к кластеру B0/W148 и все они были лекарственно устойчивы. В Абхазии, 22 из 23 изолятов В0/W148 были МЛУ, в то время как другие включенные в исследование изоляты генотипа Beijing были чувствительны к противотуберкулезным препаратам (10 из 55) \cite{pardini2009characteristics}. В Эстонии изоляты кластера B0/W148 составили 37.2\% от всех лекарственно устойчивых штаммов генотипа Beijing, в то время как ни одного чувствительного штамма B0/W148 в данном исследовании выявлено не было, включая штаммы, выделенные в 1994 году \cite{kruuner2001spread}. Также следует отметить, что в исследовании 2,092 образцов из 24 стран Европы методом VNTR кластер B0/W148 был выявлен в 470 случаях (17 стран, преимущественно Восточная Европа). Согласно исследованию этот кластер называется ECDC0002 и в крайней степени ассоциирован с лекарственной устойчивостью.
Довольно интересной является гипотеза о происхождении штаммов и первичном их распространении. По мнению Мокроусова штаммы кластера зародились в Сибири до 1960х годов, что в целом согласуется с исследованием Merker с соавт. (касательно даты возникновения). В дальнейшем, в ходе программы по освоению целины (1955-1960 годы), миграционные потоки были направлены в Казахстан и в Сибирь. Следует отметить, что в Казахстане представленность кластера В0/W148 крайне мала и составляет около 4\%. Это подтверждает гипотезу автора о том, что в европейской части России штаммов кластера в те годы еще не было. В свою очередь вторая волна миграции, 1960-1980 годы, напротив, из Сибири по всей стране могла повлечь массовое распространение представителей кластера. Согласно Мокроусову, триггером распространения именно устойчивого клона могло послужить повсеместное использование открытого в 1963 году рифампицина.
 
\subsection{Применение масс-спектрометрии в протеомике}
Протеомика исследует всю совокупность белков, синтезируемых организмом/клеткой в данной среде и на конкретном этапе клеточного цикла. Она описывает их качественный состав, относительную представленность, взаимодействие с другими макромолекулами, а также посттрансляционные модификации (ПТМ) \cite{haekkinen2000cell, molloy2002proteomics, monteoliva2004differential}. Белки играют важную роль почти во всех биологических процессах, соответственно в клетках существуют тысячи белков, каждый из которых подвергается взаимодействию, как с другими белками, так и с целыми клеточными компартментами.

Для исследования белков протеомика использует множество технических подходов. К ним относятся визуализация клеток с помощью световой и электронной микроскопии; эксперименты с кристаллами и изучением их структуры; и прочие. Другой мощный протеомный подход фокусируется на анализе \ti{de-novo} белков или совокупности белков, выделенных из клеток или тканей. Такие исследования, как правило, затруднительны из-за большой сложности клеточных протеомов и низкой представленности многих белков (а так же большого динамического диапазона концентраций белков), что требует высокочувствительных аналитических методов и приборов. Сейчас масс-спектрометрия (MS) становится основным методом для анализа сложных белковых образцов. Протеомика на основе MS - это дисциплина, основывающаяся на существовании баз данных последовательностей генов и геномов, а также техническим и концептуальным достижениям во многих областях, особенно, в создании и разработке методов ионизации \cite{aebersold2003mass}. За разработку методов идентификации и структурного анализа биологических макромолекул, и, в частности, за разработку методов масс-спектрометрического анализа биологических макромолекул, в 2002 году была вручена нобелевская премия Джону Фенну и Коити Танакака.

%\subsection{Проведение масс-спектрометрического эксперимента}
Стандартный эксперимент по исследованию белков с использованием масс-спектрометра состоит из пяти основных этапов. Схема представлена на рисунке \ref{mass_scheme}. На первом шаге белки из клеточного лизата или ткани извлекаются и очищаются. Как правило, за этим следует разделение полученной смеси гель-электрофорезом. Следующим этапом является фрагментирование белка на пептиды. Обычно это происходит при помощи фермента трипсина. На третьем этапе пептиды, предварительно разделенные жидкостной или иной хроматографией, подвергаются ионизации и поступают в масс анализатор. После MS-анализа пептиды могут подвергнуться повторной фрагментации. Новые, дочерние, ионы так же подвергаются анализу. Это пятый этап (MS/MS или тандемная масс-спектрометрия). МС-основанная протеомика зарекомендовала себя как незаменимая технология для определения закодированной в геноме информации. На сегодняшний момент белковый анализ (первичная структура, посттранскрипционные модификации или белок-белковые взаимодействия) при помощи МС является наиболее успешным при работе с небольшим (по сравнению с другими методами белкового анализа) количеством белка, выделенным из различных образцов. Системный анализ большого количества генов, экспрессированных в клетки, является основной целью протеомики; эта область сейчас быстро развивается, в основном, благодаря разработке новых экспериментальных подходов \cite{aebersold2003mass}.

\begin{figure}[ph!]
    \begin{center}
        \includegraphics[width=0.9\linewidth]{mass_scheme.png}
    \end{center}
\caption[foo bar]{Схема проведения стандартного масс-спектрометрического эксперимента. Адаптировано из \cite{aebersold2003mass}.}
\label{mass_scheme}
\end{figure}

% \subsection{Orbitrap}

\subsection{Методы протеогеномики}
Для получения протеомных данных обычно используют метод <<\fl{shotgun proteomics}>> - комбинация жидкостной хроматографии и тандемной масс-спектрометрии \cite{bantscheff2012quantitative}. Одним из ключевых шагов в протеомике является идентификация пептидов на основе полученных MS/MS спектров. В отличие от геномных технологий, вроде ДНК или РНК секвенирования, где происходит непосредственное "прочитывание" исходной последовательности, в протеомике, как правило, пептиды идентифицируются за счет сопоставления экспериментальных MS/MS спектров и теоретических спектров всех пептидов, представленных в базе, против которой осуществляется поиск \cite{nesvizhskii2010survey}.
При таком поиске используются следующие исходные предположения: 
\begin{inparaenum}
    \item все белок-кодирующие последовательности генома точно известны и аннотированы
    \item все эти последовательности включены в базу, против которой осуществляется поиск
\end{inparaenum}.
Весь последующий анализ, включая идентификацию, количественный анализ и прочий статистический анализ, основывается на этих предположениях \cite{nesvizhskii2005interpretation}.

Проблема такого подхода заключается в том, что не все пептиды представлены в текущей поисковой базе или какой-либо другой. Пептиды могут содержать мутации, находиться в новых генах, перед неверно аннотированным стартом или в альтернативных сплайсформах. 
Один из способов идентификации пептидов с мутациями заключается в масс-тег подходе. При этом подходе происходит идентификация коротких участков пептида, после чего осуществляется поиск в более широком диапазоне масс прекурсеров \cite{dasari2010tagrecon}. 

Более общим подходом является протеогеномика. Термин впервые был использован в 2004 и изначально применялся в исследовании, где протеомные данные использовались для улучшения качества аннотации \cite{jaffe2004proteogenomic}. С тех пор этот термин используется в более общем смысле. В протеогеномном подходе, пептиды идентифицируются за счет идентификации MS/MS спектров против специальной базы, включающей в себя последовательности новых предсказанных белков, а так же различные варианты последовательностей уже аннотированных белков. Такие базы получаются за счет использования геномной и транскриптомной информации. Таким образом, протеогеномика позволяет не только подтвердить текущую аннотацию, но так же уточнить её \cite{nesvizhskii2014proteogenomics}. 

\subsubsection{Типы пептидов, идентифицируемых при протеогеномном исследовании}
Пептиды, идентифицируемые при протеогеномном поиске, соответствуют различным участкам генома. Такие пептиды можно разделить на межгенные (находятся между аннотированными генами) и внутригенные (находятся полностью или частично в областях, где содержится аннотированный ген). Внутригенные можно разделить на 
\begin{inparaenum}
    \item находящиеся в белок-кодирующих генах 
    \item находящиеся в длинных некодирующих РНК
    \item находящиеся в псевдогенах
\end{inparaenum} \cite{harrow2012gencode}. Большинство пептидов, идентифицируемых в протеогеномике, уже известны и относятся к аннотированным генам. В эукариотах (в которых присутствует интроно-экзоная структура) большинство пептидов относятся к экзомам, и, как правило, меньше 20\% относятся к экзон-экзон участкам. Новые пептиды, не идентифицируемые против какой-либо базы, могут находиться в неаннотированных участках генома, быть результатами единичной аминокислотной замены (SAP), находиться в нетранслируемых регионах (3' или 5' UTR) или интронах, являться результатом альтернативного сплайсинга \cite{nesvizhskii2014proteogenomics}.

\subsubsection{Подходы к созданию поисковых баз}
Идентификация пептидов против специальных баз является ключевым шагом в протеогеномике. Обычно база состоит из известных аннотированных и предсказанных последовательностей. При протеогеномном поиске следует внимательно относиться к размеру базы: увеличение размера влечет за собой увлечение времени поиска и FDR. Оптимальный выбор зависит от того, что требуется в эксперименте: точность или чувствительность \cite{nesvizhskii2014proteogenomics}. Рассмотрим наиболее распространенные варианты создания баз для протеогеномного поиска.

\textbf{Транслирование в шести рамках генома} - такая база может быть получена в результате транслирования в шести рамках генома \cite{baerenfaller2008genome}. Недостатком такого подхода является большой размер итоговой базы (в основном состоящей из последовательностей несуществующих белков) и невозможности поиска экзом-экзом пептидов, в случае эукариот. Например транслированный таким образом геном человека приводит к базе в ~3.2 гигабазы белковых последовательностей, что в 70 раз больше, чем референс в 45 мегабаз \cite{khatun2013whole}. Для уменьшения размера базы могут применены различные вычислительные методы: выбор последовательностей, имеющих гомологии с уже известными белками; использование методов предсказания кодирующего потенциала; исключение слишком коротких последовательностей (например, меньших, чем 30 аминокислот) \cite{blakeley2012addressing}. 

%\textbf{Ab initio предсказание генов} 

%\textbf{Expressed sequence tag (EST) data}

\textbf{Аннотированные РНК-транскрипты} Белковые последовательности могут быть получены в результате шестирамочного транслирования аннотированных РНК-транскриптов, например из баз Enselbl или RefSeq. Это позволяет идентифицировать альтернативные сайты инициации трансляции. База GENCODE содержит 84408 мРНК аннотированных белков. В результате транслирования такой базы получается белковая база в 200 мегабаз, что всего в 4.5 раза больше референса \cite{khatun2013whole}. Так же такие базы могут содержать последовательность, аннотированные как псевдогены или длинные некодирующие РНК \cite{derrien2012gencode}.

%\textbf{RNA-seq данные}

\textbf{Различные вариации последовательностей} Белковые последовательности в рефернсной базе могут быть дополнены последовательностями, являющимися вариациями референсных последовательностей (как правило, это SAP, делеции и инсерции). Для каждой вариации берется большая область вокруг неё и добавляется в базу, как отдельная, независимая последовательность. Одно аминокислотные замены можно скачать из различных баз данных: \fl{NCBI dbSNP, Online Mendelian Inheritance, Protein Mutant Database} \cite{li2011bioinformatics}.  

%\textbf{Прочие специализированные базы}

\subsubsection{Влияние размера базы}
Возможность идентифицировать спектры пептидов, полученные в результате MS/MS экспериментов, используя поисковые базы данных, зависит от многих факторов. Во-первых, пептид должен присутствовать в поисковой базе. Однако, чем больше сравнивается теоретических спектров с экспериментальным, тем больше вероятность того, что лучший результат будет у неверного теоретического спектра, и тем труднее различать верные и неверные идентификации \cite{nesvizhskii2010survey}. В результате, поиск против большой базы может дать несколько новых идентификаций белков или пептидов, но при этом общее количество идентификаций будет меньше, в сравнении с поиском против референсного сиквенса \cite{blakeley2012addressing, krug2013deep}. Так же увеличение базы приводит к увеличению машинного времени, необходимого для этого поиска. Таким образом, одним из ключевых моментов в протеогеномике является поиск баланса между размером базы и её содержимым.

\subsubsection{Способы увеличения чувствительности идентификации пептидов}
Протеогеномные базы, как правило, больше соответствующих белковых баз, используемых в обычных экспериментах. Чтобы компенсировать эту разницу можно применять различные методы и техники. Подходы, используемые в протеомике и протеогеномике для увеличения числа идентификаций, можно разделить на две группы: технические (связанные с подготовкой и непосредственно проведением масс-спектрометрического эксперимента) и вычислительные (связанные с различными подходами обрабоки и валидации полученных результатов). В качестве вычислительного подхода можно рассматривать идентификацию с помощью нескольких поисковых машин одного и того же набора данных \cite{shteynberg2013combining}. После поиска происходит изменение скорингов идентификации пептидов за счет комбинации нескольких источников информации, используя подходы машинного обучения \cite{nesvizhskii2010survey}. Например, такую функцию выполняет Scaffold, позволяющий объединить результаты поиска нескольких поисковых машин. Одной из дополнительных стратегий для сокращения пространства поиска является фракционирование смеси пептидов, проводимое до \fl{LS-MS/MS} анализа. Фракционирование, например, может проходить за счет определенных физико-химических свойств пептида или свойств последовательности пептида. Примером фракционирования может служить изоэлектрическая фокусировка. Спектры, полученные от фракции с определенной изоэлектрической точкой, можно искать против пептидов с примерно такой же теоретически предсказанной изоэлектрической точкой \cite{branca2014hirief}. Другим примером фракционирования может служить разделение геля на несколько полос, в результате чего происходит разделение исходной смеси пептидов. Независимое измерение каждой фракции позволяет снимать спектры низко-представленных пептидов, которые не были бы отобраны на MS-2, в случае, если бы фракционирование не проводилось \cite{bespyatykh2016proteome}.

Чувствительность идентификации так же можно повысить за счет многоступенчатого анализа данных. На первом этапе можно проводить поиск против \textquotedblleftэталонной\textquotedblright\ базы, наиболее точно описывающей исследуемый организм и позволяющей идентифицировать большинство спектров. На втором этапе происходит поиск против расширенной базы для идентификации дополнительных спектров \cite{ning2010computational, helmy2012mass}. При таком подходе результаты первоначального поиска используются для уточнения второй базы, используемой при дальнейшем анализе. 

\subsubsection{Точность идентификации}
Одной из целей протеогеномики является поиск новых областей генома, кодирующих белки, и уточнение уже аннотированных. Поэтому требуется дополнительная точность идентификаций, чтобы предотвратить внесение ошибок в аннотацию. Ошибки могут возникать и накапливаться практически на любом этапе работы. В протеогеномике, как и в протеомике, для предотвращения накопления ошибок при переходе с PSM на уровень уникальных пептидов, избыточные PSM должны быть свернуты в один PSM с наивысшим скорингом \cite{nesvizhskii2010survey}. Пептиды, идентифицированные в различных состояниях (например, двух- или трех-зарядные ионы; модифицированные и не модифицированные формы) так же должны быть объединены или обрабатываться вероятностно, с учетом удельных весов \cite{shteynberg2011iprophet}. 

При использовании многоэтапного поиска, когда результаты первой идентификации используются при подготовки базы для дальнейшей, более специфической идентификации, необходимо на каждом шагу добавлять соответствующие количество decoy-последовательностей в базу \cite{nesvizhskii2010survey}. 

Одним из способов отслеживания количества ложных идентификаций служит FDR, позволяющий оценить долю ложно-положительных идентификаций. Выделяют глобальный FDR, характеризующий общую долю ложно положительных идентификаций во всём эксперименте, и локальный, характеризующий вероятность ложной идентификации каждого отдельного пептида или белка. Кроме контроля глобального FDR, используемого в большинстве работ по протеомике, необходимо следить за достоверностью каждой отдельной идентификации (например, при глобальном пептидном FDR в 5\% ошибка идентификации отдельных пептидов может превышать это значение). Вероятность идентификации белка или события (новый ген, альтернативный старт, сплайс форма, в случае протеогеномики, может быть рассчитана на основании вероятностей соответствующих уникальных пептидов \cite{shteynberg2011iprophet, castellana2010proteogenomics}.

%\subsubsection{Класс-специфический анализ идентификаций и FDR}


\subsubsection{Ложная неслучайная идентификация пептидов}
Ошибочная идентификация пептида происходит в одном из двух случаев: либо произошло случайное совпадение с высоким скором MS/MS-спектра и несвязанного с ним пептидом из базы, либо в результате гомологии последовательности пептида из базы и истинной последовательности пептида. Вне зависимости от типа используемой при поиске decoy-базы (например реверсивное или случайное прочтение исходных последовательностей), ложные идентификации второго типа, скорей всего, будут недооценены \cite{nesvizhskii2010survey}. Часто ложная идентификация происходит в результате химической модификации высоко представленного пептида, если в результате сдвига масс из-за модификации, масса пептида становится эквивалента массе некоторого другого пептида из базы \cite{nesvizhskii2006dynamic, abraham2013moving}. Для исключения таких идентификаций можно проверить, например с использованием BLAST, схожесть последовательности каждого нового идентифицированного пептида, с последовательностями всех пептидов, представленных в референсной базе, и исключить (или отдельно контролировать) все высоко гомологичные. Если для дальнейшего анализа требуется сохранить такие пептиды (например при поиске одно аминокислотных замен), нужно проверить, что наблюдаемая разница масс между новым и референсынм пептидом не совпадает с массой какой-либо распространенной химической или пост-трансляционной модификацией \cite{li2011bioinformatics}. Список наиболее частных модификаций, специфичных для исследуемого биологического объекта, можно получить, используя 'blind' поисковый алгоритм \cite{tsur2005identification}. Кроме того, замена лейцина на изолейцин и обратная не может быть идентифицирована с помощью масс-спектрометрии. Пептиды, содержащие такие замены, должны быть исключены из дальнейшего анализа.

\subsubsection{Интерпретация данных и новых событий}
В протеомике результатом идентификации является список идентифицированных белков или генов, а так же список идентифицированных пептидов, с определенным уровнем FDR. В протеогеномике к таким результатам, так же добавляются списки новых событий, вроде "новый ген"\,, "новый кодирующий регион"\,, "альтернативный старт"\  и так далее, с соответствующими пептидами, подтверждающие эти события. Пептиды с одинаковыми последовательностями могут быть получены из различных областей генома (например, от паралогов, идентичных сайтов разных белков или из псевдогенов). Такие пептиды не могут быть доказательством экспрессии с какого-то определенного участка генома \cite{nesvizhskii2005interpretation}. Кроме того, новые пептиды могут быть интерпретированы разными способами, например как новый транскрипт этого гена или как пептид из интрона или нетранслируемого региона того же гена. 

\subsubsection{Идентификация новых пептидов}
Одним из результатов протеогеномного анализа является список новых пептидов. Этот список зависит от выбора референсной базы и её версии. Как обсуждалось ранее, для многих организмов существуют несколько версий белковых баз, и эти базы отличаются объемом и качественном аннотации. Более того, эти базы регулярно обновляются, в результате в них добавляются и удаляются последовательности. Таким образом, в протеогеномным исследованиях, пептиды идентифицируемые против специфичных баз, должны быть проверены на вхождение в основанные базы, существующие для данного организма.

\subsection{Применение протеогеномики}

\subsubsection{Поиск новых белок-кодирующих регионов}
Возможность применение масс-спектрометрических протеомных данных для поиска новых белок-кодирующих регионов и подтверждения границ уже аннотированных, обсуждалась начиная с первых дней существования протеомики, как науки в её современном виде \cite{choudhary2001interrogating, andersen2001mass}. Такие результаты чаще всего достигаются за счет поиска против специальных баз данных, полученных в результате шести рамочного транслирования генома, трех рамочного транслирования предсказанных различными методами белок-кодирующих участков генома или шести (трех, в случае стрнед-специфичного секвенирования) рамочного транслирования данных РНК-секвенирования. Как правило, новые белок-кодирующие регионы находят у новых, немодельных организмов \cite{castellana2014automated, castellana2008discovery, yang2011discovery}. Даже для хорошо исследованых эукариот есть работы, в которых находят новые гены. Например, исследование белкового профиля человека и мыши с использованием шестирамочного транслирования, позволило идентифицировать 98 и 52, соответственно, ранее не аннотированных белок-кодирующих областей \cite{branca2014hirief}.

\subsection{Идентификация коротких рамок считывания и сайтов инициации трансляции}
Предсказание коротких рамок считывания (sORF) и рамок, начинающихся не с канонических старт-кодонов, особенно затруднительно \cite{yang2011discovery}. В работе Yang и соавторов было показано, что короткие рамки могут составлять до 10\% от всех белок-кодирующих элементов в геноме эукариот \cite{yang2011discovery}. Хотя с развитием методов рибопрофилирования и РНК-секвенирования появилась возможность находить sORF, протеогеномика позволяет получить дополнительное подтверждение белок-кодирующего потенциала этих sORF \cite{menschaert2013deep, oyama2007diversity, slavoff2013peptidomic}. Поиск пептидов, подтверждающих sORF и сайты инициации трансляции, будет более эффективным, если провести дополнительные этапы подготовки образца. Для повышения вероятности идентификации белков из sOFR можно использовать фракционирование, перед LC-MS/MS экспериментом \cite{slavoff2013peptidomic}. Для обнаружения новых сайтов инициации трансляции, как правило, требуется точная идентификация N-концевого пептида, которые могут быть обогащены перед масс-спектрометрическим анализом за счет использования N-концевых меток \cite{hartmann2014n}. 


\subsection{Протеогеномика \ti{M. tuberculosis}}
\subsubsection{\ti{M. tuberculosis} H37Rv}
Для проверки и корректировки аннотации Kelkar и соавторы провели протеогеномное исследование \ti{M.tuberculosis} H37Rv с использованием масс-спектрометрических данных \cite{kelkar2011proteogenomic}. В своей работе, в качестве геномной базы, авторы использовали транслированный в шести рамках геном. В качестве стартовых кодонов использовались GTG и TTG, которые транслируются как метионин, а не валин и лейцин соответственно, в случае, если они являются стартовыми сайтами \cite{cole1998erratum}. 

GSSP пептиды были разделены на 3 группы:
\begin{inparaenum}
    \item относящиеся к межгенным областям
    \item частично пересекающиеся с аннотированным геномом
    \item полностью относящиеся к аннотированным генам
\end{inparaenum}.

Они обнаружили 41 новый ген, и корректировку рамки для 79 генов. Из 79 генов с измененной рамкой: для 78 было исправлено положение стартового кодона, и два гена были объединены в один. Для подтверждения этих результатов использовались альтернативные программы для аннотации генома (\fl{FgeneSB, GeneMark 2.5}), а так же поиск гомологий среди известных генов, используя алгоритм BLAST. 

Авторы работы были удивлены, что они обнаружили новые гены с учетом того, что геном был секвенирован более 10 лет назад, и аннотирован множеством независимых групп.

\subsubsection{\ti{Mycobacterium smegmatis}}
Похожее исследование было проведено Potgieter и соавторами для \ti{Mycobacterium smegmatis}. В своей работе они использовали две геномные базы: одну для поиска новых генов, другую для идентификации TSS. 

Первая база была получена в результате транслирования в шести рамках геном бактерии. В качестве стартовых кодонов использовались ATG, GTG, и TTG. Исключались последовательности, длина которых была меньще 20 аминокислот. В результате получилась база в 79481 последовательностей. При поиске против этой базы было идентифицированною 77 ORF, не содержащих аннотированных генов, по 1 пептиду, и 44 ORF по двум и более пептидам. Для ORF, идентифицированных только по одному пептиду была проведена дополнительная проверка: выполнялось выравнивание с использованием BLAST против известных протеомных и транскриптомных баз. В результате 19 ORF были отобраны, как "низкоуровневые доказательства"\ для изменения аннотации генома.

Вторая база использовалась для поиска TSS. Она была получена в результате \ti{ab inito} предсказания белок-кодирующих генов, с использованием программы GeneMarkS. Так же эта база подтверждала новые гены и изменения в уже аннотированных. Размер базы составил 6655 последовательностей. При поиске против этой базы было идентифицировано 137 не триптических TSS-пептидов и 547 триптических TSS-пептидов. Так же было идентифицировано 3 ORF с альтернативными сайтами начала трансляции \cite{potgieter2016proteogenomic}. 



\newpage
