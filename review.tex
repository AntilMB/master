\section{Литературный обзор}

\subsection{Mycobacterium tuberculosis}

\subsection{Применение масс-спекртрометрии в протеомике}

\subsection{Orbitrap}

\subsection{Количественная протеомика}


%  ##          ###    ########  ######## ##          ######## ########  ######## ########    
%  ##         ## ##   ##     ## ##       ##          ##       ##     ## ##       ##          
%  ##        ##   ##  ##     ## ##       ##          ##       ##     ## ##       ##          
%  ##       ##     ## ########  ######   ##          ######   ########  ######   ######      
%  ##       ######### ##     ## ##       ##          ##       ##   ##   ##       ##          
%  ##       ##     ## ##     ## ##       ##          ##       ##    ##  ##       ##          
%  ######## ##     ## ########  ######## ########    ##       ##     ## ######## ########


\subsection{Безметочная квантификация}
Вне зависимости от выбранного метода безметочного количественного анализа, эксперимент должен включать в себя следующие шаги:
\begin{inparaenum}
    \item  пробоподготовка, включая извлечение белков, их очистку, трипсинолиз и прочие шаги 
    \item разделение пептидов при помощи различных хроматографических методов с 
    последующий MS/MS анализом 
    \item анализ полученных результатов: идентификация, количественный анализ и статистический анализ
\end{inparaenum}
В целом, методы безметочного количественного анализа можно разделить на две групы: на основе интенсивности ионов или за счет spectral counting \cite{zhu2009mass}.

\subsubsection{Относительная квантификация по интенсивности MS-1 сигнала}
При ионизации электроспреем, интесивность MS-1 иона коррелирует с его концентрацией \cite{voyksner1999investigating}. 
Впернвые количественный анализ белков и пептидов за счет интенсивности LC-MS пиков был проведен на миоглобине. Были проанализированы концентрации в диапазоне от 10 фемтамоль до 100 пикомоль.
После 

\subsection{Коилчественный анализ по количеству спектров}
Квантификация по спектральному показателю основана на сравнении количества идентифицированных тандемных спектров для одного и того же белка из различных проб. Луи и соавторы изучали связь между относительной концентрацией белка и следующими тремя факторами: процентом идентифицированной (покрытой) части белка, количеством наблюдаемых пептидов и спектральным показателем. Было показано, что среди этих факторов только спектральный показатель имеет сильную линейную корреляцию с относительной концентрацией белка (коэффициент детерменации r2 = 0.9997), в динамическом диапазоне, составляющем два порядка. Таким образом, спектральный показатель можно использовать для определения относительной представленности белка в данном образце. Зайбелов и соавторы провели исследование мембранных белков S. cerevisiae. Они провели идентификацию и количественный анализ, используя подход спектрального показателя и мечением 15N. Была показана высокая корреляция между двумя методами для белков, с высоким отношением “сигнал/шум”. Так же было установлено, что метод спектрального показателя имеет больший  динамический диапазон и воспроизводимость.

\newpage
