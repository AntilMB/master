\section{Результаты и обсуждение}

\subsection{Подготовка и проведение масс-спектрометрических измерений}
Тут что-нибудь про бактерий напиать.
Были отсняты масс-спектры 30 культур. Из них 6 были не фракционированы, остальные были разделены на 6 фракций. Разделение на фракции позволило идентифицировать пептиды, которые не были бы идентифицированы при обычном поиске, из-за того, что на измерения MS/MS спектра были бы отобраны более высоко представленные пептиды. Количество отснятых спектров для каждой культуры представлены в таблице.

\subsection{Подготовка баз}
Скаченная с NCBI аннотация содержала 4103 аннотированных белок-кодирующих последовательностей. После добавления контаминант и decoy-последовательностей получилась белковая база, состоящая из 8258 последовательностей. После транслирования генома в 6 рамках и исключения коротких последовательностей получилось 74488 белок-кодирующих последовательностей. В результате добавления контаминант и decoy-последовательностей, получилась геномная база, состоящая из 149028 последовательностей. Исключение коротких последовательностей из геномной базы позволило ускорить поиск; снизить пороговые значения идентификации, тем самым повысив чувствительность подхода. Минимальная длинна аннотированного белка \ti{W-148} составляет 84 аминокислоты, таким образом, исключение рамок длинной меньше чем 33 аминокислоты не должно привести к потерям при идентификации.

\subsection{Идентификация белков и пептидов}
Поиск проходил при помощи поисковых маших Mascot и X!Tandem. Объединение результатов поиска и перерасчет FDR был произведен в Scaffold. Против белковой базы было идентифицировано 32054 пептида (1041059 psm), против геномной базы 36502 уникальных пептида (1131085 psm). Пересечение идентифицированных пептидов представленно на рисунке После вычитания результатов поиска против белковой базы из результатов поиска против геномной базы получилось 6015 GSSP (Genome search specific peptides).  
                                                


\newpage
