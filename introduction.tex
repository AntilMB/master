\section{Введение}
\ti{Mycobacterium tuberculosis} является возбудителем тяжелой болезни - туберкулеза, наносящей большой вред здоровью. Особенно остро проблема стоит в развивающихся странах. Появление резистентных и более вирулентных штаммов усугубило ситуацию. Было сделано множество работ по исследованию протеома этого патогена \cite{jungblut1999comparative, mattow2006proteins}. Есть несколько работ, посвященных аннотации генома \ti{M.tubeculosis}. Полногеномное секвинирование штамма \ti{Mycobacterium tuberculosis H37Rv} было проведено в 1998 году, затем был отсекверирован штамм \ti{CDC1551} и несколько других \cite{cole1998erratum, fleischmann2002whole}. Точная аннотация белок-кодирующих генов является постоянно меняющимся и модернизирующимся техническим процессом. Это особено заметно в случае \ti{M.tubeculosis}. В работе Коула и его колег, было показано наличие 3924 ORF в геноме \ti{H37Rv} \cite{cole1998erratum}. При повторной аннотации, проведенной теми же авторами, число генов стало 3995 \cite{camus2002re}. В марте 2011 база \fl{TubercuList} содеражала 4012 аннотированых белок-кодирующих генов. де Соуза с коллегами провели сравнение двух различных аннотаций для \ti{H37Rv}, сделанных группами \fl{Sanger, TIGR}, и обнаружили, что 50\% генов имеют различные стартовые сайты \cite{de2008high}. В этой же работе, используя протеомные данные 449 культур, авторы смогли исправить аннотацию 24 генов. Так же, возможность существования CDSs, которые ещё не были аннотированы в \ti{H37Rv}, были предложены в работе Лью и коллег \cite{lew2011tuberculist}.

\newpage
