\section{Введение}
С древнейших времен и по наше время туберкулез остается серьезным инфекционным заболеванием, от которого ежегодно умирают более трех миллионов человек. Его инфекционным началом являются микобактерии туберкулезного комплекса, включающего \ti{Mycobacterium tuberculosis}, \ti{M. bovis}, \ti{M. africanum}, \ti{M. canettii}, \ti{M. microti}, \ti{M. pinnipedii} и \ti{M. caprae} и другие \cite{brosch2002new}. Долгое время считалось, что геномы этих видов исключительно консервативны. Микобактерий туберкулёзного комплекса характеризует отсутствие выраженного горизонтального переноса в процессе их эволюции, которая имеет выраженный «клональный» характер. Тем не менее, к настоящему времени использование секвенирования геномов микобактерий и их отдельных генов, применение разнообразных методов определения однонуклеотидного, делеционного и микросателлитного полиморфизма показало, что микобактерии туберкулеза более полиморфны, чем предполагалось ранее \cite{tsolaki2004functional}. На данный момент выделяют 7 основных филогенетических линий: lineage 1, lineage 2, lineage 3, lineage 4, lineage 5, lineage 6, lineage 7. Среди них lineage  2, в основном представленная генотипом Beijing, является одной из самых распространенных клад в современной всемирной эпидемии туберкулеза. При этом семейство Beijing является наиболее встречаемым среди новых случаев заболевания туберкулезом в разных регионах Российской Федерации и странах ближнего зарубежья. В свою очередь даже среди представителей этого семейства выявляют более «успешные» кластеры. Одним из таких вариантов является кластер Beijing B0/W148. Штаммы, относящиеся к нему, обладают повышенной устойчивостью к противотуберкулезным препаратам, а также более выраженной вирулентностью.
В последние годы становится очевидно, что для реализации успешной стратегии борьбы с заболеванием необходимо системное исследование организации \ti{M. tuberculosis} на молекулярном уровне с привлечением различных “-омиксных” технологий. При этом геномная информация является неким шаблоном клетки, напрямую или косвенно кодирующим все остальные молекулы. Реализация данной информации происходит через синтез РНК (транскриптом) и белков (протеом). И если транскриптом является первым, очень динамичным и функциональным шагом, то протеом отражает сумму всех процессов, происходящих в клетке. Полная последовательность первого генома \ti{M. tuberculosis} кластера В0/W148 была опубликована в виде единого скаффолда в 2011 году (штамм W-148; GenBank номер ACSX00000000.1.). В конце 2015 года геном был полностью отсеквенирован и аннотирован (штамм W-148; GenBank CP012090.1), а в виде чтений с секвенаторов следующего поколения в базе данных NCBI представлены сотни штаммов кластера. Это послужило чрезвычайно полезным достижением для более глубокого изучения штаммов кластера, однако так и не дало ответов на многие биологические вопросы. Одна из причин связана с тем, что большое количество белков аннотации представлено в виде гипотетических белков с неизвестной функцией. В свою очередь даже сама аннотация является динамической и относительно хорошо исследована только для референсного штамма H37Rv, относящегося к филогенетической линии 4.  Так, например, в работе Cole и коллег было показано наличие 3924 ORF в геноме H37Rv \cite{cole1998erratum}. При повторной аннотации, проведенной теми же авторами, число генов составило 3995 \cite{camus2002re}. В марте 2011 база \fl{TubercuList} содержала 4012 аннотированных белок-кодирующих генов. De Souza с коллегами провели сравнение двух различных аннотаций для H37Rv, сделанных группами \fl{Sanger} и \fl{TIGR}, и обнаружили, что 50\% генов имеют различные стартовые сайты \cite{de2008high}. В этой же работе, используя протеомные данные о 449 фильтрованных из культуры белков, авторы смогли исправить аннотацию 24 генов. Также была предложена возможность существования CDSs, которые ещё не были аннотированы в \ti{H37Rv}, в работе Lew J и коллег \cite{lew2011tuberculist}.
Перечисленные исследования относятся к области протеомики, называемой протеогеномика. Данное направление позволяет проводить поиск новых белков и соответствующих им открытых рамок считывания, а также корректировать уже имеющиеся ORF. Проведение анализа такого рода на эпидемиологически важном объекте может привести к нахождению ранее неописанных особенностей жизнедеятельности патогена и тем самым пролить свет на понимание механизмов “успешности” штаммов кластера.

%Белки играют ключевую роль в функционировании организма, так как могут выполнять самые разные функции: от регуляторных, до строительных. Молекулы белков состоят из одной или нескольких полипептидных цепей, организованных в характерную трёхмерную структуру. Индивидуальные белки имеют определённый химический состав. Их молекулярные массы охватывают интервал от 6000 до более миллиона дальтон 1. Так Фридрих Энгельс дал следующее определение: “Жизнь есть способ существования белковых тел”.

%\ti{Mycobacterium tuberculosis} является возбудителем тяжелой болезни - туберкулеза -, наносящей большой вред здоровью. Особенно остро проблема стоит в развивающихся странах. Появление резистентных и более вирулентных штаммов усугубило ситуацию. Было сделано множество работ по исследованию протеома этого патогена \cite{jungblut1999comparative, mattow2006proteins}. Есть несколько работ, посвященных аннотации генома \ti{M.tubeculosis}. Полногеномное секвенирование штамма \ti{Mycobacterium tuberculosis} H37Rv было проведено в 1998 году, затем был отсеквенирован штамм CDC1551 и несколько других \cite{cole1998erratum, fleischmann2002whole}. Точная аннотация белок-кодирующих генов является постоянно меняющимся и модернизирующимся техническим процессом. Это особенно заметно в случае \ti{M.tubeculosis}. В работе Коула и его коллег, было показано наличие 3924 ORF в геноме H37Rv \cite{cole1998erratum}. При повторной аннотации, проведенной теми же авторами, число генов стало 3995 \cite{camus2002re}. В марте 2011 база \fl{TubercuList} содержала 4012 аннотированных белок-кодирующих генов. де Соуза с коллегами провели сравнение двух различных аннотаций для H37Rv, сделанных группами \fl{Sanger, TIGR}, и обнаружили, что 50\% генов имеют различные стартовые сайты \cite{de2008high}. В этой же работе, используя протеомные данные 449 культур, авторы смогли исправить аннотацию 24 генов. Также возможность существования CDSs, которые ещё не были аннотированы в \ti{H37Rv}, была предложена в работе Лью и коллег \cite{lew2011tuberculist}. \ti{Mycobacterium tuberculosis} H37Rv является модельным организмом, которое изучает большое количество научных групп. 

%На территории России большое распространение получил штамм W-148, отличающийся от H37Rv большой вирулентностью и лекарственной устойчивостью. При этом механизмы, вызывающие эти эффекты неизвестны. Поэтому задача поиска новых генов и изменений уже аннотированых является актуальной. 

\newpage

\section{Цели и задачи}
\subsection{Цель работы}
    Провести протеогеномный анализ штамма \ti{M. tuberculosis} W-148 

\subsection{Задачи работы}

\begin{enumerate} 
    \item Сформировать выборку данных масс-спектрометрических измерений для штамма W-148
    \item Подготовка поисковых баз c последующей идентификацией спектров
    \item Интерпретировать результаты идентификации:
    \begin{itemize}
        \item Найти новые, неаннотированные гены
        \item Провести корректировку аннотации старт-кодона
        \item Проверить экспрессию псевдогенов
    \end{itemize} 
    \item Провести валидацию полученных результатов:
    \begin{itemize}
        \item Время выхода и ошибки масс при идентификации пептидов
        \item Потенциальная остаточная контаминация на приборе
        \item Интерпретация модифицированной аминокислоты, как другой немодифицированной
    \end{itemize}
    
\end{enumerate}

\newpage