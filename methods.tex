\section{Материалы и методы}

\subsection{Получение бактерий}

\subsection{Проведение масс-спектрометрического эксперимента}

\subsection{Контроль качества}


\subsection{Протеогеномика}
\subsubsection{Создание поисковых баз}
В работе использовалось 2 типа баз: белковая и геномная. Белковая база - аннотированные последовательности, для данного штамма. Геноманя - база, полученная в результате транслирования генома в шести рамках.
Белковые базы для \ti{M.tuberculosis W-148} и \ti{M.tuberculosis H37Rv} были составлены из аннотированных белков штаммов (NCBI Reference Sequence: NZ\_CP012090.1, 4244 аминокислотных последовательностей для \ti{W-148} и ).
Геномные базы были получены в результате 6 рамочного транслирования от стоп- до стоп-каднона геномов штаммов \ti{M.tuberculosis W-148} и \ti{M.tuberculosis H37Rv}, используя программу \fl{Artemis} версия 16.0.0 \cite{rutherford2000artemis}. При транслировании использовалась 11 трансляционная таблица \fl{NCBI}. Минимальная длинна рамки была установлена в 100 нуклеиновых кислот.
К каждой базе были добавлены последовательности 26 контаминантных белков (кератины, альбумины, трипин).

\subsubsection{Идентификация новых белков}

\subsubsection{Уточнение N-конца}

\subsubsection{Анализ SAP}

\subsection{Идентификация пептидов и белков}
% написать общий план идентифкации про 2 машины + мердж результатов
Данные полученные в результате LC-MS/MS эксперимента (Raw формат) были сконвертированы в пик-лист (MGF формат), используя \fl{ProteoWizard msconvert} \cite{chambers2012cross}. Идентификация проходила против двух белковых и двух геномных баз с использованием \fl{Mascot Search Engine version 2.5.1} \cite{cottrell1999probability}. Параметры поиска были следующими: триптические пептиды, не более двух пропущенных сайтов трипсинолиза, ошибка массы прекурсера 20 ppm, ошибка массы фрагментов 0.5 Да, заряды прекурсера 2+, 3+, 4+. \fl{Oxidation(M), Carbamidomethylation(C) and Deamidated(NQ)} были устанолвены как возможнные модификации пептидов. Для подсчета FDR и порогового скоринга использовался поиск против decoy-базы, полученной в результате реверса исходной базы. FDR был выбран на уровне 5\%.
Пептид считался идентифицированным, если его скор выше порогового скоринга и ранг равен еденице. Белок считался идентифицированным, если для него нашлось два и более уникальых пептидов.

\subsection{Количественный анализ}
\subsubsection{Анализ в MaxQuant}
\subsubsection{Анализ в Progenesis LC-MS}
\subsubsection{Сравнение результатов программ}



\newpage
