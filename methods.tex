\section{Материалы и методы}

\subsection{Получение бактерий}

\subsection{Проведение масс-спектрометрического эксперимента}

\subsection{Контроль качества}
Для всех масс-спектров был проведен контроль качества масс-спектров с использованием программного решения реализованного в лаборатории биоинформатики НИИ ФХМ. В ходе контроля качества были проверены следующие факторы: качество трипсинолиза, распределение зарядов родительских ионов, ошибка измерения m/z для родительских и дочерних ионов, распределение идентифицированных пептидов по времени удерживания пептидов в хроматографической колонке.

\subsection{Создание поисковых баз}
В работе использовалось 2 типа баз: белковая и геномная. Белковая база - аннотированные последовательности, для данного штамма. Геноманя - база, полученная в результате транслирования генома в шести рамках.
Белковые базы для \ti{M.tuberculosis W-148} и \ti{M.tuberculosis H37Rv} были составлены из аннотированных белков штаммов (\ti{W-148}: \fl{NCBI Reference Sequence: NZ\_CP012090.1}, версия от 11 марта 2017 года, 4103 аминокислотных последовательности, 137 псевдогена; \ti{H37Rv}: \fl{NCBI Reference Sequence: NC\_000962.3}, версия от 2 августа 2016 года, 3932 аминокислотных последовательности).
Геномные базы были получены в результате 6 рамочного транслирования от стоп- до стоп-каднона геномов штаммов \ti{M.tuberculosis W-148} и \ti{M.tuberculosis H37Rv}, используя программу \fl{Artemis} версия 16.0.0 \cite{rutherford2000artemis}. При транслировании использовалась 11 трансляционная таблица \fl{NCBI}. Минимальная длинна рамки была установлена в 100 нуклеиновых кислот.
К каждой базе были добавлены последовательности 26 контаминантных белков (кератины, альбумины, трипин) и decoy-последовательности, полученные в результате прочтения аминокислотных последовательностей с конца, за исключением стартового метеонина. 

\subsection{Идентификация пептидов и белков}
% написать общий план идентифкации про 2 машины + мердж результатов
Данные полученные в результате LC-MS/MS эксперимента (Raw формат) были сконвертированы в пик-лист (MGF формат), используя \fl{ProteoWizard msconvert} \cite{chambers2012cross}. Идентификация проходила против двух белковых и двух геномных баз с использованием \fl{Mascot Search Engine version 2.5.1} \cite{cottrell1999probability} и \fl{X!Tandem version 3.4.3} \cite{fenyo2003method}. Результаты идентификации двух программ были объеденины в \fl{Scaffold version 4.2.1}. 

Параметры поиска \fl{Mascot} были следующими: триптические пептиды, не более одного пропущенного сайта трипсинолиза, ошибка массы прекурсера 20 ppm, ошибка массы фрагментов 0.04 Да, заряды прекурсера 2+, 3+, 4+. \fl{Oxidation(M)} была устанолвена как возможнная модификация пептидов, \fl{Carbamidomethylation(C)} как фиксированная. 

Параметры \fl{X!Tandem} были следующими: триптические пептиды, не более одного пропущенного сайта трипсинолиза, ошибка массы прекурсера 20 ppm, ошибка массы фрагментов 50 ppm, проверка не моноизотопных масс, \fl{Carbamidomethylation(C)} - фиксированная модификация, \fl{Oxidation(M)} возможная модификация.

Результаты работы поисковых машин были объединины в \fl{Scaffold} с параметрами: 1:1 \fl{forward/decoy ratio}, \fl{LFDR scoring}, стандартные белковые группы, не проводить GO-аннотацию. Белковый и пептидный FDR был установлен на уровне 1\%, 1 и более пептидов на белок. Результаты были экспортированны в виде листа идентифицированных пептидов.

\subsection{Протеогеномика \ti{W-148}}
Координаты аннотированных генов были пересечены с учетом стренда и фрейма с координатами ORF, полученными в результате шестирамочного транслирования.
Для поиска GSSP из результатов поиска против геномной базы \ti{W-148} были исключены пептиды, идентифицированные против белковой базы \ti{W-148}. Так же были исключены пептиды идентифицируемые против геномной базы и предсталвенные в аннотации, так же пептиды идентифицируемые только в одном эксперименте. Для дальнейшего анализа были выбраны ORF, в которых произошло одно из следующих событий:
\begin{inparaenum}
    \item идентифицированно два и более уникальных GSSP-пептидов
    \item идентифицированн GSSP-пептид и присутствует аннотированный ген
    \item идентифицированн GSSP-пептид и есть пересечение по координатам с псевдогеном в пределах стренда
\end{inparaenum}.

\subsubsection{Идентификация новых белков}
Рассматривались ORF, в которых было идентифицированно два и более уникальных GSSP-пептида. Для проверки потенциала кодирующей способности рамки, был проведён blastp против базы nr. 

\subsubsection{Уточнение N-концов}

\subsection{Сравнение идентификаций против \ti{W-148} и \ti{H37Rv}}

\subsubsection{Поиск новый генов}

\subsubsection{Уточнение N-концов}

\subsection{Визуалиция данных}
Для визуализации данных использовался \fl{Gbrowse}. Были выделены следующие группы: 
\begin{inparaenum}
    \item аннотированные гены
    \item псевдогены
    \item ORF с новыми генами
    \item ORF с пептидами, идентифицируемые перед аннотированным стартом
    \item GSSP-пептиды
\end{inparaenum}. Для этого результаты идентификации были обработаны и экспортированны в gff3 формате.


\newpage
