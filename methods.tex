\section{Материалы и методы}

\subsection{Бактериальные штаммы}
Коллекция клинических образцов включала в себя 63 изолята \ti{M. tuberculosis} кластера B0/W148, полученных от пациентов, проходивших лечение в поликлинике Санкт-Петербургского НИИ Фтизиопульмонологии с 2005 по 2015 год. Идентификацию, выделение чистой культуры патогена и пробоподготовку для масс-спектрометрического исследования проводили в специализированной лаборатории представленного НИИ и в лаборатории молекулярной генетики микроорганизмов ФГБУ ФНКЦ ФХМ ФМБА России \cite{bespyatykh2016proteome}. 

\subsection{Проведение масс-спектрометрического эксперимента}
Для протеогеномного анализа использовались масс-спектры белковых клеточных лизатов для \ti{M. tuberculosis}, полученных с прибора \fl{AB SCIEX TripleTOF 5600} в лаборатории протеомного анализа ФГБУ ФНКЦ ФХМ ФМБА России и \fl{ThermoFisher Q Exactive Hybrid Quadrupole-Orbitrap} в ФГБУ ИБМХ имени В.Н. Ореховича. Часть сырых данных была загружена в архив ProteomeXchange Consortium \cite{vizcaino2014proteomexchange} (PXD005478).

\subsection{Контроль качества}
Для всех масс-спектров был проведен контроль качества масс-спектров с использованием программного решения, реализованного в лаборатории биоинформатики ФНКЦ ФХМ. В ходе контроля качества были проверены следующие факторы: качество трипсинолиза, распределение зарядов родительских ионов, ошибка измерения m/z для родительских и дочерних ионов, распределение идентифицированных пептидов по времени удерживания пептидов в хроматографической колонке.

\subsection{Создание поисковых баз}
В работе использовалось 2 типа баз: белковая и геномная. Белковая база - аннотированные последовательности, для данного штамма. Геномная - база, полученная в результате транслирования генома в шести рамках.
Белковые базы для \ti{M.tuberculosis W-148} были составлены из аннотированных белков штаммов (\ti{W-148}: \fl{NCBI Reference Sequence: NZ\_CP012090.1}, версия от 11 марта 2017 года, 4103 аминокислотных последовательности, 137 псевдогена).
Геномная база была получена в результате 6 рамочного транслирования от стоп- до стоп-кодона генома штамма \ti{M. tuberculosis} W-148, используя программу \fl{Artemis} версия 16.0.0 \cite{rutherford2000artemis}. При транслировании использовалась 11 трансляционная таблица \fl{NCBI}. Минимальная длинна рамки была установлена в 100 нуклеиновых кислот. Пример транслированного в шести рамках участка генома представлен на рисунке \ref{6frame_base}.
К каждой базе были добавлены последовательности 26 контаминантных белков (кератины, альбумины, трипин) и decoy-последовательности, полученные в результате прочтения аминокислотных последовательностей с конца, за исключением стартового метионина. 

\begin{figure}[h!]
    \begin{center}
        \includegraphics[width=1\linewidth]{6frame_base.png}
    \end{center}
\caption[foo bar]{Транслированный в шести рамках геном. По центру обозначены геномные координаты. Сверху - 1-3 фрейм, снизу 4-6 фреймы. Вертикальными линиями обозначены стоп-кодоны. Голубым - ORF, длинной более 100 пар оснований.}
\label{6frame_base}
\end{figure}

\subsection{Идентификация пептидов и белков}
Данные, полученные в результате LC-MS/MS эксперимента (Raw формат), были преобразованы в пик-лист (MGF формат), используя \fl{ProteoWizard msconvert} \cite{chambers2012cross}. Идентификация проходила против двух белковых и двух геномных баз с использованием \fl{Mascot Search Engine version 2.5.1} \cite{cottrell1999probability} и \fl{X!Tandem version 3.4.3} \cite{fenyo2003method}. Результаты идентификации двух программ были объединены в \fl{Scaffold version 4.2.1}. 

Параметры поиска \fl{Mascot} были следующими: триптические пептиды, не более одного пропущенного сайта трипсинолиза, ошибка массы прекурсера 20 ppm, ошибка массы фрагментов 0.04 Да, заряды прекурсера 2+, 3+, 4+. \fl{Oxidation(M)} была установлена как возможная модификация пептидов, \fl{Carbamidomethylation(C)} как фиксированная. 

Параметры \fl{X!Tandem} были следующими: триптические пептиды, не более одного пропущенного сайта трипсинолиза, ошибка массы прекурсера 20 ppm, ошибка массы фрагментов 50 ppm, проверка не моноизотопных масс, \fl{Carbamidomethylation(C)} - фиксированная модификация, \fl{Oxidation(M)} возможная модификация.

Результаты работы поисковых машин были объединены в \fl{Scaffold} с параметрами: 1:1 \fl{forward/decoy ratio}, \fl{LFDR scoring}, стандартные белковые группы, не проводить GO-аннотацию. Белковый и пептидный FDR был установлен на уровне 1\%, 1 и более пептидов на белок. Результаты были экспортированны в виде листа идентифицированных пептидов.

\subsection{Протеогеномика \ti{W-148}}
Координаты аннотированных генов были пересечены с учетом цепи и рамки с координатами ORF, полученными в результате шестирамочного транслирования.
Для поиска GSSP из результатов поиска против геномной базы \ti{W-148} были исключены пептиды, идентифицированные против белковой базы \ti{W-148}. Так же были исключены пептиды идентифицируемые против геномной базы и представленные в аннотации, и пептиды, идентифицируемые только в одной культуре. Для дальнейшего анализа были выбраны ORF, в которых произошло одно из следующих событий:
\begin{inparaenum}
    \item идентифицировано два и более уникальных GSSP
    \item идентифицирован GSSP и присутствует аннотированный ген
    \item идентифицирован GSSP и есть пересечение по координатам с псевдогеном в пределах цепи
\end{inparaenum}.

\subsubsection{Валидация результатов идентификации}
Для каждого GSSP была проведена проверка времени выхода и ошибки масс при идентификации; потенциальной остаточной контаминации на приборе; ошибки интерпретации модифицированной аминокислоты, как другой немодифицированной.

Проверка ошибки идентификации масс проходила на уровне PSM. В результатах запуска масс-спектрометра, для каждого PSM, соответствующему GSSP, находилось среднее значение и стандартное отклонение ошибки идентификации всех PSM в диапазоне +/- 5 минут от времени выхода данного PSM. Из дальнейшего анализа исключались все PSM, ошибка идентификации масс которых отличалась более чем на 3 стандартных отклонения от средней ошибки в установленном временном интервале. В каждом ране были отфильтрованы 10\% самых ранних и поздних по времени выхода PSM.

Для проверки времени выхода результаты идентификации были разделены на две группы: PSM, относящиеся к идентифицированным против белковой базы, и PSM, относящиеся к GSSP. Для каждого пептида была вычислена его гидрофобность, используя библиотеку qqman языка R \cite{turner2014qqman}. По PSM, относящиеся к белковой идентификации была построена линейная регрессия, где в качестве зависимой переменной использовалась время выхода, а независимой - гидрофобность. Полученная модель была применена к PSM, относящимся к GSSP. 10\% пептидов, показавших наибольшее отклонение были исключены из дальнейшего анализа.

Был проведен поиск точного и с учетом одной возможной замены вхождения GSSP в другие белки, представленные в базе \fl{NCBInr}.  

\subsubsection{Идентификация новых белков}
Рассматривались ORF, в которых было идентифицировано два и более уникальных GSSP-пептида, прошедших все этапы валидации, и в которых не содержится аннотированный ген. Для проверки потенциала кодирующей способности рамки был проведён поиск гомологов в базе NCBInr, с использованием алгоритма BLAST. 

\subsubsection{Уточнение N-концов}
Рассматривались ORF, в которых было идентифицировано два и более уникальных GSSP-пептида и в которых содержится аннотированный ген. Новые рамки сравнивались с аналогичными генами в \ti{H37Rv}.

%  \subsection{Сравнение идентификаций против \ti{W-148} и \ti{H37Rv}}
%  \subsubsection{Поиск новый генов}
%  \subsubsection{Уточнение N-концов}

\subsection{Визуалиция данных}
Для визуализации данных использовался \fl{Gbrowse}. Были выделены следующие глифы: 
\begin{inparaenum}
    \item аннотированные гены
    \item идентифицированные пептиды
    \item псевдогены
    \item ORF с новыми генами
    \item ORF с пептидами, идентифицируемые перед аннотированным стартом
    \item GSSP-пептиды.
\end{inparaenum} Результаты идентификации были обработаны и экспортированы в gff3 формате.


\newpage

%\subsection{Тест шрифта с бакалавром}
%пути внутри клетки. Подобная информация может быть применена в медицинских целях. 

%Для количественного анализа белка можно использовать различные методы, например Гель-электрофорез, капиллярный электрофорез, жидкостную хроматографию, различные оптические и спектрографические измерения, центрифугирование, масс-спектрометрию и множество других. Наиболее эффективным методом для количественного анализа сложной смести белков является комбинация жидкостной хроматографии (или высоко эффективной жидкостной хроматографии) с масс-спектрометрией (или тандемной масс-спектрометрии). Это позволяет устанавливать количественный состав смеси не только при помощи сложного процесса меченья белков, но и посредством безметочного анализа состава смеси, как относительного, так и абсолютного. 

%Melioribacter resous - недавно открытая, умеренно теплолюбивая, факультативно анаэробная, хемоорганотрофическая бактерия . Её штамм P3M-2 был получен из микробной плёнки, развивающийся в жёлобе, покрытом древесной корой, находящимся под потоком горячей воды (температура 46 o C), выходящей из поисково-разведочной нефтяной скважины глубиной в 2775 метров, находящейся недалеко от Томска в России. Метаболически универсальная бактерия, способная расти за счёт питания сахарами или пептидами, за счёт аэробного дыхания, нитратов или Fe(III). Недавний анализ генома выделил ключевые элементы цепи переноса электронов, позволяющий понять способность бактерии адаптироваться к изменяющимся условиям окружающей среды, за счёт комбинации окисления различных доноров электронов при аэробном дыхании или сокращения акцепторов электронов 5 .
