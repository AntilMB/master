\newpage

\section{Выводы}

\begin{enumerate} 
\item Применение нескольких поисковых машин, с последующим объединением результатов, позволяет увеличить количество и качество идентификаций.

\item При обработке большого количества данных использование белкового FDR позволяет сократить количество ложно-положительные идентификаций белков, так как этот критерий становится более весомым, чем пептидный.

\item В результате протеогеномного анализа штамма \ti{Mycobacterium tuberculosis} W-148 были идентифицированы 16 новых генов и у 249 (24) открытых рамок считывания были скорректированы старты трансляции.

\item Разработано программное решение для протеогеномного анализа, которое можно за короткое время перенести на любой прокариотический организм.
\end{enumerate}
\newpage
